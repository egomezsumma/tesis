\documentclass[11pt,a4paper,twoside]{tesis}
% SI NO PENSAS IMPRIMIRLO EN FORMATO LIBRO PODES USAR
%\documentclass[11pt,a4paper]{tesis}

\usepackage{graphicx}
\usepackage[utf8]{inputenc}
\usepackage[english]{babel}
\usepackage[left=3cm,right=3cm,bottom=3.5cm,top=3.5cm]{geometry}
\usepackage{natbib}
\begin{document}

%%%% CARATULA
% Comentar y descomentar según corresponda
%\def\titulo{Licenciada }
\def\titulo{Licenciado }

\def\autor{Leonel Exequiel G\'omez}
\def\tituloTesis{La Guerra de las Galaxias: \mbox{Rebelión e Imperio}}
\def\runtitulo{La Guerra de las Galaxias: Rebelión e Imperio}
\def\runtitle{Star Wars: Rebellion and Empire}
\def\director{Obi-Wan Kenobi}
\def\codirector{Master Yoda}
\def\lugar{Buenos Aires, 2011}
\input{caratula}

%%%% ABSTRACTS, AGRADECIMIENTOS Y DEDICATORIA
\frontmatter
\pagestyle{empty}
\input{abs_esp.tex}

\cleardoublepage
\input{abs_en.tex}

\cleardoublepage
\input{agradecimientos.tex} % OPCIONAL: comentar si no se quiere

\cleardoublepage
\input{dedicatoria.tex}  % OPCIONAL: comentar si no se quiere

\cleardoublepage
\tableofcontents

\mainmatter
\pagestyle{headings}

%%%% ACA VA EL CONTENIDO DE LA TESIS


\chapter{Introducci\'on}
\section{Imágenes por resonancia magnética (MRI)}
%{\begin{small}%
%\begin{flushright}%
%\it
%There's nothing for me now.
%I want to learn the ways of\\ the Force and become a Jedi like my father. \\
%--Luke Skywalker
%\end{flushright}%
%\end{small}%
%\vspace{.5cm}}


%ESTR:
%[MRI zoom:3][DMRI zoom3]


La Resonancia Magn\'etica es una t\'ecnica útil para generar im\'agenes de 
\'organos internos del cuerpo, sin la necesidad de someter al mismo a un 
procedimiento quir\'urgico. Estas im\'agenes permiten a los m\'edicos hacer distintos tipos de 
diagn\'osticos. La imagen construida por esta t\'ecnica esta fuertemente 
relacionada con la densidad de part\'iculas de hidr\'ogeno en un determinado 
voxel (p\'ixel en tres dimensiones). Dicha densidad es representada 
con una intensidad dentro de la escala crom\'atica.

Esta t\'ecnica, a grandes rasgos, se basa en los siguientes fen\'omenos f\'isicos. El primero de 
ellos, es 
una propiedad que poseen las part\'iculas de los n\'ucleos de hidr\'ogeno, 
llamada momento angular intrínseco (\textit{spin angular momentun} en ingles). 
Dichas part\'iculas se encuentran girando todo el tiempo en su 
propio eje a una frecuencia determinada. La razón de su  momento 
angular intrínseco se la llama razón giromagnética (\textit{gyromagnetic 
ratio} en ingles) y lo denotamos con el símbolo $\gamma$. El valor de esta 
razón es propio de cada sustancia y su unidad en el 
\textit{sistema internacional} es de radianes sobre segundos por tesla. Debido 
a esto, dichos n\'ucleos se encuentran 
magnetizados y se comportan como un peque\~no im\'an bipolar, los cuales 
propagan cierto campo 
magn\'etico en una cierta direcci\'on. Es por ello, que podemos representarlos como un vector en 
tres dimensiones. La orientaci\'on del campo la representamos con la direcci\'on del vector y la 
magnitud del campo con longitud del mismo.

El siguiente fen\'omeno, es la consecuencia de aplicar un campo 
magnético $B$ a partículas que se encuentran girando en su propio 
eje. La consecuencia será que estas comenzarán a describir un movimiento de 
precesi\'on alrededor de un eje. Dicho eje se encuentra alineado en la misma 
direcci\'on que el campo magn\'etico $B$. La frecuencia de este movimiento, en 
cada part\'icula será proporcional a $B$ y a su razón giromagnética.

\begin{center}
$w=\gamma  B $    \textit{[ecuación de Larmor]}
\end{center}


Medir el campo magn\'etico de una \'unica part\'icula (o inclusive de varias) resulta imposible. 
Debido a que, por un lado, el campo magn\'etico es muy d\'ebil como para poder ser captado por las 
maquinarias de hoy en d\'ia. Y la distribuci\'on uniforme de las part\'iculas 
de 
hidr\'ogeno en los tejido provoca que sus respectivos vectores se cancelen (es decir, que sumen una 
resultante nula). 

Con el fin de poder medir el campo magnético que propagan 
las partículas, se les aplican una secuencia de campos magn\'eticos gradientes 
(es decir, que poseen distinta intensidad a lo largo de una direcci\'on). 
Dichos campos magn\'eticos gradientes hacen que las part\'iculas se alineen 
convenientemente y que 
a distintas posiciones en el espacio, posean distintas frecuencias de 
precesi\'on. La posibilidad de 
alinearlas causa que la resultante de los vectores que representan a las part\'iculas ya no se 
cancelen. Y que giren a distintas frecuencias en funci\'on de la posici\'on, 
ayuda a discriminar de que punto o plano del cuerpo proviene la se\~nal que 
queremos medir.

Ya que podemos saber la frecuencia en la que giran las part\'iculas en cierta posici\'on (por la
ecuación de Larmor). Y gracias al fen\'omeno f\'isico conocido como 
\textit{resonancia}, podemos generar un pulso radio magnético el cual 
incrementar\'a considerablemente la velocidad de giro a las part\'iculas que 
est\'en girando a la misma frecuencia que dicho pulso. Excitar un conjunto 
de part\'iculas de esta manera en una cierta posici\'on, les aplica una 
gran cantidad de energ\'ia. Y ahora si, con esta energía adicional, el campo 
magn\'etico que propagan los n\'ucleos de hidr\'ogeno excitados puede ser 
captado por una m\'aquina. Con esta se\~nal los escáners pueden construir las 
im\'agenes MRI.


\section{Resonancia magnética de difusi\'on}


La resonancia magnética de difusi\'on (dMRI del termino en ingles 
\textit{Diffusion MRI}) es la extensi\'on de MRI con el fin de poder adem\'as  
representar la 
difusi\'on de las mol\'eculas del agua dentro de los tejidos. En nuestro caso particularmente 
dentro 
de la materia blanca del cerebro. La importancia cl\'inica de las im\'agenes de difusi\'on cobro 
importancia cuando \citet{Moseley1990} reportaron que con dichas im\'agenes es 
posible discriminar entre tejido normal y tejido isqu\'emico. Ninguna 
otra t\'ecnica pod\'ia hacerlo, con la excepci\'on de la tomograf\'ia con 
emisi\'on de positrones. 

Uno de los desafíos importantes de la física y de la química del siglo $xx$, 
fue como cuantificar la difusión de un liquido en un medio. En 1950 
\citet{Hahn1950} se dio cuenta la señal MRI era sensible a la difusión.

Fue as\'i, que \citet{CarrH.Y.andPurcell1954} tuvieron la idea de sensibilizar 
los escáners de resonancia magnética al movimiento de las part\'iculas de agua, 
modificando la secuencia de campos magn\'eticos gradientes.

Como la difusi\'on de las part\'iculas de agua es influenciada por la estructura del medio en el 
cual difunde, entonces la resonancia magnética puede usarse para entender la 
estructura de los tejidos de una manera no invasiva.  


Casi una d\'ecada despu\'es \citet{Stejskal1965} proponen una nueva secuencia 
llamada \textit{Pulse Gradient Spin Echo} (PSGE). La cual agrega otro
gradiente a la secuencia utilizada para generar las imágenes MRI. Este nuevo 
gradiente $G$ a\~nade velocidad de precesi\'on a los n\'ucleos en un 
determinado momento del experimento por un tiempo $\delta$. Luego se apaga dicho 
campo magn\'etico gradiente y se enciende otro exactamente igual, pero en 
sentido contrario y durante el mismo per\'iodo de tiempo $\delta$. El primer 
gradiente esta causando que los n\'ucleos que est\'an precediendo se desfacen. 
Desfasarse en este caso provoca que una componente de la resultante de los 
vectores que representan a los n\'ucleos, se cancele, con lo 
cual el escáner percibir\'a una ca\'ida de se\~nal mucho m\'as r\'apida que con 
el experimento sin gradiente $G$. Ahora bien, el segundo gradiente, debería  
corregir esa diferencia de fase (ya que le 
aplica el mismo impulso, el mismo tiempo, pero en el sentido contrario). Sin embargo no es del todo 
cierto, y es porque las part\'iculas no est\'an quietas. Las mismas se est\'an moviendo debido a la 
difusi\'on. Entonces esa caída más abrupta de la señal nos 
aporta informaci\'on de la difusión en ese punto. Es por ello, que para 
hablar de difusi\'on lo que hacemos es comparar una 
imagen sin gradientes $G$ y una con un gradiente no nulo $G$. Porque la 
diferencia que hay entre 
esas dos mediciones es debido a la difusi\'on.


\begin{center}
$E(G, \Delta, \delta) = \frac{S(G, \Delta, \delta)}{S_0}$ 
$$S_0=S(q=0)$$
\end{center}

A $E(G, \Delta, \delta)$ la llamamos \textit{atenuaci\'on de la señal} y 
$\Delta$ es el tiempo entre la activación del primer gradiente y el segundo en 
sentido contrario, el cual describimos en el párrafo anterior. Esta señal nos 
habla 
\'unicamente de la difusi\'on de las part\'iculas de hidr\'ogeno en la misma 
direcci\'on en la cual 
se aplico el campo magn\'etico gradiente $G$, en un voxel determinado del 
cerebro. Dicha $G$ es un vector que representa la intensidad 
del campo gradiente y su direcci\'on.

Cabe aclarar que la estructura del cerebro es $anisiotropica$ en relación 
a la difusión de las part\'iculas hidrógeno. Esto quiere decir, que el valor 
medido en un determinado voxel depende de la direcci\'on en la que se la mida. 
Es por ello, que para reconstruir los modelos que 
representan la difusi\'on de las partículas de hidrógeno en el cerebro, es 
necesario hacer mediciones en varias direcciones distintas. Inclusive, existen 
modelos como NODDI (ref) que adem\'as de distintas direcciones precisan 
distintas intensidades de G. 

\begin{verbatim}
Le bihan introducir q
\end{verbatim}




Tomar varias mediciones para distintos valores de gradientes, se lo llama 
muestrear el \textit{q-space}, ya que en cierta forma se esta 
muestreando la señal para varios valores la variable $q$, siendo que $G$ esta codificada en dicha 
variable.


Con el correr de los años una de las lineas de investigación más 
importante en el \'area fue la interpretaci\'on f\'isica de la se\~nal recibida 
por los escáners de MR.  



\section{Modelos de difusi\'on}
La materia blanca es uno de los varios tipos de tejidos que encontramos en el cerebro. Se sabe que 
la misma esta conformada en su mayor\'ia por axones \citep{Purves2004}. Los axones son una 
prolongaci\'on larga que constituye la morfolog\'ia de las c\'elulas neuronales. 
Luego entender en que direcciones prefiere difundir el agua dentro de la materia 
blanca es \'util para poder entender la estructura de los 
axones con un estudio \textit{in-vivo} (ref). Dado que, aunque las part\'iculas 
de hidr\'ogeno pueden difundir hacia cualquier direcci\'on del espacio, su gran mayor\'ia 
encontrara 
menos obst\'aculos al difundir en una direcci\'on alineada con el ax\'on (o fajo de axones) que se 
disponga en ese lugar. 
Como la difusi\'on es una propiedad anisiotropica, para representarla no basta un escalar. Precisamos un objeto 
matem\'atico conocido como \textit{tensor}, que sirve para representar magnitudes independientemente del espacio
vectorial en el que se lo mida. Es as\'i que \citet{Basser1994} propone medir la 
atenuaci\'on de la se\~nal en varias direcciones y as\'i aproximar el tensor de difusi\'on 
$D$.[falta]
  
[Mencionar que hacen los modelos y listar un par]\\%


[Introducir Tractograf\'ia]



\section{IQT}
[problema de la resoluci\'on]
En MRI la resoluci\'on t\'ipica de las im\'agenes es del orden de los mil\'imetros, mientras las 
escalas de las longitudes en los tejidos neuronales son del orden de los micrones. La t\'ipica 
resoluci\'on de los voxeles de hoy en d\'ia es de $2\times2\times2 mm^3$. Esta diferencia en 
escala, provoca groseros errores en los resultados de las tractografias. Es por ello, que es de 
suma importancia mejorar la resoluci\'on de las im\'agenes de disfusi\'on. Reducir el volumen del 
voxel es desafiante porque trae aparejado una mayor relaci\'on se\~nal ruido en las mediciones. 

Para apalear este problema se han propuestos varios m\'etodos para obtener im\'agenes de alta 
resoluci\'on. Los cuales caen en dos grandes categor\'ias. El primero de ellos toman como 
entrada una sola imagen en baja resoluci\'on y por medio de t\'ecnicas de 
\textit{interpolaci\'on} obtienen una imagen de mayor resoluci\'on. El segundo grupo, m\'as 
conocido como \textit{super-resolution}, toma como entrada varias im\'agenes en baja resoluci\'on y 
con ellas intenta construir una en alta resoluci\'on. [falta]









% TODO
%[MRI zoom 0]
%spins+frecuencia de larmor+gradientes+resonancia $\rightarrow$ voltios en una antena receptora
%se\~nal recibida+fft $\rightarrow$ imagen 

%[dMRI zoom 0]
%e(q) = s(q)/S0 $\rightarrow$ lo unico que pudo haber cambiado es por culpa de la difusion


%[dMRI bueno para + enfermedades]
% Diffusion MRI (dMRI) is therefore the method of choice to probe microstructure, because it is 
%sensitive to the micron-scale displacement of water molecules, and is therefore strongly affected 
%by 
%the number, orientation and permeability of barriers (e.g. myelin) and the presence of various 
%cell 
%types and organelles (e.g. neurons, dendrites, axons, neurofilaments and microtubules) in living 
%tissue (Beaulieu, 2002). In part\'icular, dMRI can detect microstructural changes in the white 
%matter 
%related to myelination and demyelination, pruning, axonal loss, and has, for this reason, become 
%particularly useful for assessing damage in white matter pathologies (Horsfield and Jones, 2002).







\chapter{La guerra de las galaxias}
\section{Infancia y juventud}
{\begin{small}%
\begin{flushright}%
\it
There's nothing for me now.
I want to learn the ways of\\ the Force and become a Jedi like my father. \\
--Luke Skywalker
\end{flushright}%
\end{small}%
\vspace{.5cm}}

\section{Rescate de la princesa}
{\begin{small}%
\begin{flushright}%
\it
Here's where the fun begins!\\
--Han Solo
\end{flushright}%
\end{small}%
\vspace{.5cm}}

\section{Sacrificio y victoria}
{\begin{small}%
\begin{flushright}%
\it
This will be a day long remembered.\\ It has seen the end of Kenobi.\\ It will soon see the end of the Rebellion.\\
--Darth Vader
\end{flushright}%
\end{small}%
\vspace{.5cm}}

%% ...
\chapter{El imperio contraataca}
\chapter{El regreso del Jedi}

%%%% BIBLIOGRAFIA
\backmatter
\bibliographystyle{plainnat}
%\bibliography{/home/routeatlas/Documentos/cscomputacion/inria/tesis/latex/references.bib}
%\bibliography{/user/lgomez/home/Documents/tesis/latex/references2.bib}
\bibliography{references.bib}


\end{document}
