\documentclass[11pt,a4paper,twoside]{tesis}
% SI NO PENSAS IMPRIMIRLO EN FORMATO LIBRO PODES USAR
%\documentclass[11pt,a4paper]{tesis}

\usepackage{graphicx}
\usepackage[utf8]{inputenc}
\usepackage[english]{babel}
\usepackage[left=3cm,right=3cm,bottom=3.5cm,top=3.5cm]{geometry}
\usepackage{natbib}
\usepackage{amsmath}
\usepackage{float}

\begin{document}

%%%% CARATULA
% Comentar y descomentar según corresponda
%\def\titulo{Licenciada }
\def\titulo{Licenciado }

\def\autor{Leonel Exequiel G\'omez}
\def\tituloTesis{La Guerra de las Galaxias: \mbox{Rebelión e Imperio}}
\def\runtitulo{La Guerra de las Galaxias: Rebelión e Imperio}
\def\runtitle{Star Wars: Rebellion and Empire}
\def\director{Obi-Wan Kenobi}
\def\codirector{Master Yoda}
\def\lugar{Buenos Aires, 2011}
\input{caratula}

%%%% ABSTRACTS, AGRADECIMIENTOS Y DEDICATORIA
\frontmatter
\pagestyle{empty}
\input{abs_esp.tex}

\cleardoublepage
\input{abs_en.tex}

\cleardoublepage
\input{agradecimientos.tex} % OPCIONAL: comentar si no se quiere

\cleardoublepage
\input{dedicatoria.tex}  % OPCIONAL: comentar si no se quiere

\cleardoublepage
\tableofcontents

\mainmatter
\pagestyle{headings}

%%%% ACA VA EL CONTENIDO DE LA TESIS

\chapter{Introducci\'on}

\section{Imágenes por resonancia magnética (MRI)}
%{\begin{small}%
%\begin{flushright}%
%\it
%There's nothing for me now.
%I want to learn the ways of\\ the Force and become a Jedi like my father. \\
%--Luke Skywalker
%\end{flushright}%
%\end{small}%
%\vspace{.5cm}}


%ESTR:
%[MRI zoom:3][DMRI zoom3]


La Resonancia Magn\'etica es una t\'ecnica útil para generar im\'agenes de 
\'organos internos del cuerpo, sin la necesidad de someter al mismo a un 
procedimiento quir\'urgico. Estas im\'agenes permiten a los m\'edicos hacer distintos tipos de 
diagn\'osticos. La imagen construida por esta t\'ecnica esta fuertemente 
relacionada con la densidad de part\'iculas de hidr\'ogeno en un determinado 
voxel (p\'ixel en tres dimensiones).

Esta t\'ecnica, a grandes rasgos, se basa en los siguientes fen\'omenos f\'isicos. El primero de 
ellos, es 
una propiedad que poseen las part\'iculas de los n\'ucleos de hidr\'ogeno, 
llamada momento angular intrínseco (\textit{spin angular momentun} en ingles). 
Mediante la cual las part\'iculas se encuentran girando en su 
propio eje a una frecuencia determinada. La razón de este movimiento se llama razón 
giromagnética (\textit{gyromagnetic 
ratio} en ingles) y lo denotamos con el símbolo $\gamma$. El valor de esta 
razón es propio de cada sustancia y su unidad en el 
\textit{sistema internacional} es de radianes sobre segundos por tesla. Debido 
a este movimiento, los n\'ucleos se encuentran 
magnetizados y se comportan como un peque\~no im\'an bipolar, los cuales 
propagan cierto campo magn\'etico en una cierta direcci\'on. Es por ello, que se los pueden 
representar como un vector en tres dimensiones. La orientaci\'on del campo la representamos con 
la direcci\'on del vector y la magnitud del campo con longitud del mismo.

El siguiente fen\'omeno, es la consecuencia de aplicar un campo magnético externo $B$ a partículas que se 
encuentran girando en su propio eje. La consecuencia será que estas comenzarán a describir un 
movimiento de precesi\'on alrededor de un eje imaginario paralelo a la direcci\'on del campo 
magnético $B$. La frecuencia de este movimiento, en cada part\'icula será proporcional a $B$ y a su 
razón giromagnética (ecuación \ref{eq:1}).

\begin{equation} 
\label{eq:1}
 w=\gamma  B
\end{equation}


Medir el campo magn\'etico de una \'unica part\'icula (o inclusive de varias) resulta imposible. 
Debido a que, por un lado, el campo magn\'etico es muy d\'ebil como para poder ser captado por las 
maquinarias de hoy en d\'ia. Y la distribuci\'on uniforme de las part\'iculas 
de hidr\'ogeno en los tejido provoca que sus respectivos vectores se cancelen (es decir, que sumen 
una resultante nula). 

Con el fin de poder medir el campo magnético que propagan las partículas, se les aplican una 
secuencia de pulsos de radiofrecuencia junto con campos magnéticos gradientes (i.e. que cambian 
linealmente en el espacio a lo largo de una direcci\'on). Dichos campos magn\'eticos gradientes 
hacen que las part\'iculas se alineen convenientemente y que a distintas posiciones en el espacio, 
posean distintas frecuencias de precesi\'on. La posibilidad de alinearlas causa que la resultante de 
los vectores que representan a las part\'iculas ya no se cancelen. Al girar a distintas 
frecuencias en funci\'on de la posici\'on, ayuda a discriminar de que punto o plano del cuerpo 
proviene la se\~nal que queremos medir.

Ya que podemos saber la frecuencia en la que giran las part\'iculas en cierta posici\'on por la
ecuación de Larmor (ecuación \ref{eq:1}). Y gracias al fen\'omeno f\'isico conocido como 
\textit{resonancia}, podemos generar un pulso radio magnético el cual 
incrementar\'a considerablemente la velocidad de giro a las part\'iculas que 
est\'en girando a la misma frecuencia que dicho pulso. Excitar un conjunto 
de part\'iculas de esta manera en una cierta posici\'on, les aplica una 
gran cantidad de energ\'ia. Y ahora si, con esta energía adicional, el campo 
magn\'etico que propagan los n\'ucleos de hidr\'ogeno excitados puede ser 
captado por una m\'aquina. Con esta se\~nal los escáneres pueden construir las 
im\'agenes MRI.


\section{Resonancia magnética de difusi\'on}


La resonancia magnética de difusi\'on (dMRI del termino en ingles 
\textit{Diffusion MRI}) es la extensi\'on de MRI con el fin de poder adem\'as  
representar la 
difusi\'on de las mol\'eculas del agua dentro de los tejidos. En nuestro caso particularmente 
dentro de la materia blanca del cerebro. Estas im\'agenes proveen valiosa información sobre la 
micro-estructura de los diferentes tipos de tejidos del cerebro (LeBihan and breton 1985). En la 
actualidad, es la única técnica capaz de evaluar 
%de forma no invasiva 
la conectividad neuronal en un cerebro humano vivo y por lo tanto posee un enorme potencial para 
resolver problemas actuales en el área de la neurociencia.

La importancia cl\'inica de las im\'agenes de difusi\'on cobro 
importancia cuando \citet{Moseley1990} reportaron que con dichas im\'agenes es 
posible discriminar entre tejido normal y tejido isqu\'emico. Ninguna 
otra t\'ecnica pod\'ia hacerlo, con la excepci\'on de la tomograf\'ia con 
emisi\'on de positrones. 

Uno de los desafíos importantes de la física y de la química del siglo $xx$, 
fue como cuantificar la difusión de un liquido en un medio. \citet{Hahn1950} se da 
cuenta que la señal MRI era sensible a la difusión. Fue as\'i, que \citet{CarrH.Y.andPurcell1954} 
tuvieron la idea de sensibilizar los escáneres de resonancia magnética al movimiento de las 
part\'iculas de agua, 
modificando la secuencia de campos magn\'eticos gradientes.

Como la difusi\'on de las part\'iculas de agua es influenciada por la estructura del medio en el 
cual difunde, entonces la resonancia magnética puede usarse para entender la 
estructura de los tejidos de una manera no invasiva.  


Casi una d\'ecada despu\'es \citet{Stejskal1965} proponen una nueva secuencia 
llamada \textit{Pulse Gradient Spin Echo} (PSGE). La cual agrega otro
gradiente a la secuencia utilizada para generar las imágenes MRI. Este nuevo 
gradiente $G$ a\~nade velocidad de precesi\'on a los n\'ucleos en un 
determinado momento del experimento por un tiempo $\delta$. Luego se apaga dicho 
campo magn\'etico gradiente y se enciende otro exactamente igual, pero en 
sentido contrario y durante el mismo per\'iodo de tiempo $\delta$. El primer 
gradiente causa que los n\'ucleos que est\'an precediendo se desfasen. 
El desfasaje provoca que una de las tres componentes que posee la resultante de los vectores que 
representan a los n\'ucleos, se cancele (por convenci\'on se coinsidera la componente vertical). Con 
lo cual el escáner percibir\'a una ca\'ida de se\~nal mucho m\'as r\'apida que con el experimento 
sin gradiente $G$. Ahora bien, el segundo gradiente, debería  corregir esa diferencia de fase (ya 
que le aplica el mismo impulso, el mismo tiempo, pero en el sentido coconsidera. Sin embargo no es 
del todo cierto, y es porque las part\'iculas no est\'an quietas. Las mismas se est\'an moviendo 
debido a la difusi\'on. Por lo tanto, la señal global, dada por la suma de los momentos magnéticos 
de todas las particulas, se atenúa debido a la incoherencia en las orientaciones de los momentos mageneticos
individuales. Entonces esa caída más abrupta de la señal nos 
aporta informaci\'on de la difusión en ese punto. Es por ello, que para 
hablar de difusi\'on lo que hacemos es comparar una 
imagen tomada sin gradientes $G$ y una con un gradiente no nulo. Porque la 
diferencia que hay entre ambas es debido a la difusi\'on.



Cabe aclarar que la estructura del cerebro es $anisiotropica$ en relación 
a la difusión de las part\'iculas hidrógeno. Esto quierepartículas el valor 
medido en un determinado voxel depende de la direcci\'on en la que se la mida. 
Es por ello, que para reconstruir los modelos que 
representan la difusi\'on de las partículas de hidrógeno en el cerebro, es 
necesario hacer mediciones en varias direcciones distintas. Inclusive, existen 
modelos como NODDI (ref) que adem\'as de distintas direcciones precisan 
distintas intensidades de G. 

El dominio de muestreo de la señal que da origen a este tipo de imágenes, no es solo 
espacial. Este comprende además la intensidad del gradiente de difusión aplicado, la dirección del mismo y el 
tiempo de difusión. Todos estos parámetros gracias al trabajo de \citet{CALLAGHAN1990177} pueden ser codificados en una única variable 
conocida como $q$ (ecuasión \refecuación Tomar varias mediciones para distintos valores de la variable $q$, se lo denomina 
muestrear el espacio $q$ (o \textit{q-space} del termino en ingles). Como esta variable esta compuesta por varios parámetros, dependiendo el tipo de experimento se 
suele hacer variar la dirección de G (a este conjunto de datos se los denomina $only$-$shell$ del termino en ingles). Otros experimentos,
además hacen variar la intensidad de G (conjunto de datos $multi$-$shell$). Y algunos mas ambiciosos, por la cantidad de datos a manejar, 
ademas varían el tiempo de difusión. Es por ello que decimos que estas imágenes además de resolución espacial, poseen resolución angular y temporal.


\begin{equation} 
\label{eq:3}
 q = \gamma \delta G
\end{equation}

\begin{equation} 
\label{eq:2}
E(q) = \frac{S(q)}{S_0} \hspace{2cm} S_0=S(q=0)
\end{equation}

%\begin{center}
%$E(G, \Delta, \delta) = \frac{S(G, \Delta, \delta)}{S_0}$ 
%$$S_0=S(q=0)$$
%\end{center}

A $E(q)$ la llamamos \textit{atenuaci\'on de la señal},  
$\Delta$ es el tiempo de difusión que es el tiempo entre la activación del primer gradiente y el segundo en 
sentido contrario, el cual describimos en el párrafo anterior. Esta señal nos 
habla \'unicamente de la difusi\'on de las part\'iculas de hidr\'ogeno en la misma 
direcci\'on en la cual se aplico el campo magn\'etico gradiente $G$, en un voxel determinado del 
cerebro. Dicha $G$ es un vector en 3 dimensiones que representa la intensidad 
del campo gradiente y su direcci\'on.


Con el correr de los años una de las lineas de investigación más 
importante en el \'area fue la interpretaci\'on f\'isica de la se\~nal recibida 
por los escáneres de MR.  



\section{Modelos de difusi\'on}
La materia blanca es uno de los varios tipos de tejidos que encontramos en el cerebro. Se sabe que 
la misma esta conformada en su mayor\'ia por axones \citep{Purves2004}. Los axones son una 
prolongaci\'on larga que constituye la morfolog\'ia de las c\'elulas neuronales. 
Luego entender en que direcciones prefiere difundir el agua dentro de la materia 
blanca es \'util para poder entender la estructura de los axones con un estudio \textit{in-vivo} 
(ref). Aunque las part\'iculas de hidr\'ogeno pueden difundir hacia cualquier 
direcci\'on del espacio, su gran mayor\'ia encontrara menos obst\'aculos al difundir en una 
direcci\'on alineada con el ax\'on (o fajo de axones) que se disponga en ese lugar. Con este 
conocimiento a priori de la anatomía de la materia blanca y para obtener detalles más finos sobre 
las características micro-estructurales del tejido cerebral, como por ejemplo el diámetro axonal 
promedio en voxel. Se puede ajustar un modelo matemático a los datos adquiridos. Uno de los 
principales objetivos de los modelos de difusión es recuperar información más detallada sobre la 
orientación de las fibras, haciendo uso del conocimiento a priori que se posee sobre la anatomía de 
los tejidos neuronales. Entre ellos se encuentra el modelo \textit{Diffusion Tensor Imaging} (DTI). 
En el cual se representa la se\~nal de difusión con un tensor de segundo orden \citep{Basser1994}. 
También existen modelos más complejos como por ejemplo MAPL \citep{Fick2016365}. El mismo pertenece 
a una familia de modelos que parte del enfoque de representar la señal de difusión con 
herramientas de análisis armónico.

\section{Problema y objetivo}
Los estándares de resolución de hoy en día de dichas imágenes, permiten hacer diversos análisis sobre conjuntos de 
fibras de mediano y gran tamaño. Sin embargo, estos análisis pierden precisión en regiones donde los grupos de axones 
son muy pequeñas en comparación al tamaño del voxel (4 ordenes de magnitud más pequeños). Consecuentemente, existe la 
necesidad de incrementar tanto la resolución espacial, angular y temporal de las imágenes dMRI. Dado que el dominio de 
muestreo de la señal difusión, no es solo espacial como ya vimos en la sección anterior. La baja relación señal ruido y el tiempo de adquisición de las 
imágenes dMRI son unos de los principales factores que limitan la resolución de las mismas. %Los limites para obtener imágenes con alta resolución 
%están relacionados con 
%el tiempo de adquisición de las mismas y la relación señal ruido (la cual decrece 
%con el tamaño voxel). 
Hoy en día, existen escáneres que pueden producir imágenes con muy alta resolución. Pero los mismos son muy costosos e 
imprácticos para escenarios clínicos. El objetivo de este trabajo es aumentar la resolución de las imágenes de difusión.






\chapter{Estado del Arte}

En MRI la resoluci\'on t\'ipica de las im\'agenes es del orden de los mil\'imetros, mientras las 
escalas de las longitudes en los tejidos neuronales son del orden de los micrones. La t\'ipica 
resoluci\'on de los voxeles de hoy en d\'ia es de $2\times2\times2 mm^3$. Esta diferencia en 
escala, provoca groseros errores en los resultados de las tractografias. Es por ello, que es de 
suma importancia mejorar la resoluci\'on de las im\'agenes de disfusi\'on. Reducir el volumen del 
voxel es desafiante porque trae aparejado una mayor relaci\'on se\~nal ruido en las mediciones. 

Debido a la escacez de escáneres con buena resolución y la imposibilidad de aplicarlos en 
escenarios 
clínicos, recientemente varias técnicas han sido propuestas para obtener imágenes con alta 
resolución. Como 
por ejemplo transferencia de calidad de imágenes (\textit{image quality transfer} en ingles). La 
cual 
consiste en explotar la 
información presente en imágenes adquiridas con alta resolución para mejorar otras con baja 
resolución 
\citep{Alexander2014}. También la técnica conocida como súper-resolución, en 
la cual se obtiene una imagen con alta resolución utilizando varias imágenes en baja resolución. 
Estas se adquieren siguiendo un esquema determinado y deben ser del mismo sujeto 
\citep{Irani1993,Robinson2010,Ning2016}. Estas técnicas pueden mejorar la resolución de las 
imágenes 
usando la 
representación de la señal de difusión, tal como la adquieren los escáners, o pueden utilizar algún 
modelo de la señal 
de difusión, de los tantos que existen en el área.
%\subsection{Otras propuestas con el mismo objetivo}
Por ejemplo en el trabajo de \citet{Alexander2014} proponen hacer transferencia de 
calidad de imágenes utilizando algoritmos de aprendizaje automático (\textit{machine learning} del 
ingles) usando los 
modelos DTI y NODDI \citep{Zhang2012}. 
%Entrenando a dicho algoritmo con un conjunto de datos de alta calidad. 
La novedad de este trabajo es que el realce se hace sobre los parámetros del modelo de difusión y 
no 
sobre la 
señal de difusión tal como la adquieren los escáners. En cambio el trabajo de \citet{Ning2016} 
combinan la t\'ecnica 
de súper-resolución con el modelo de difusión \textit{compressed-sensing} \citep{Naidoo2015}. A 
diferencia del primer 
trabajo, utilizan un algoritmo de optimización convexa para reconstruir la imagen con alta calidad. 
En ambos casos el 
algoritmo depende fuertemente del modelo seleccionado para representar la señal de difusión.



\chapter{Experimento}

En este trabajo nos proponemos combinar los conceptos de transferencia de calidad de imagenes y 
modelo MAPL. El algoritmo propuesto recontruye una imagen en alta calidad a partir de su 
equivalente en baja calidad, utilizando informaci\'on de un conjunto de imagenes con altos 
estandares de calidad. Para ello coinsideraremos la representaci\'on de la imagen con el modelo 
MAPL, el cual es util para representar se\~nales de difusi\'on funciones bases con un conjunto 
esparzo de coeficientes. Al igual que en el trabajo de \citep{Ning2016} consideraremos a la 
imagen de baja resolucion como la version en menor resoluci\'on de la imagen en alta resolucion. 
Pero a diferencia de este trabajo En la seccion \ref{mapl}.

\section{Teoria}

En esta secci\'on introduciremos brevemente el modelo MAPL que ser\'a utilizado en el algoritmo de 
reconstrucci\'on propuesto.

\subsection{MAPL}
Este modelo pertenece a la familia de modelos que representa la se\~nal de difusion como una suma 
de funciones bases. El mismo es una extension del modelo MAP-MRI (\citet{Ozarslan2013}) que mejora 
la estimacion de los coeficientes que MAP define, incorporando un regulizador Laplaceano. 

MAP-MRI se puede describir la se\~nal de difusion de la siguiente manera:

\begin{equation}
    E(q) = \sum_{i}^{N_{coef}} c_i \Phi_{N_i}(q)
\end{equation}

Donde ${c_i}$ son los coeficientes a ajustar y $\Phi_{N_i}$ son las funciones bases. Las funciones 
bases esta dada por el producto de tres funciones bases  ortogonales de una dimensi\'on de la 
siguiente manera.

$$
    \Phi_{n_1 n_2 n_3} (A, \mathbf{q}) = \phi_{n_1}(U_x, q_x) \phi_{n_2}(U_y, q_y) 
\phi_{n_3}(U_z, q_z)$$
  
con   
$$\phi_n(u, q) = \frac{i^{-n}}{\sqrt{2} n!} e^{-2\pi^2q^2u^2} H_n(2\pi u q)$$

Donde $H_n$ es el polinomio de Hermite de orden $n$. El parametro $\mathbf{q}$ es el parametro $q$ 
explicado en la introduci\'on pero rotado en la direcci\'on del mayor autovector del tensor 
estimado por el modelo DTI en ese punto (\citep{Basser1994}). La llamada matriz de 
covarianza de desplazamiento $A$ se calculada como $A=2D\tau$, con $\tau$ el tiempo de difusi\'on 
(\citep{Basser2002}). Los llamados factores de escala $(u_i)$ se obtinen como 
$A=diag(u_x^2,u_y^2,u_z^2)$. Para un orden radial $N_{rad}$ el numero de coeficiente viene dado por 
$N_{coef} = (N_{rad}+2)(N_{rad}+4)(2N_{rad}+3)/24$. Para comprender mejor los indices para 
construir las funciones bases que define MAP veremos un ejemplo con $N_{rad}=2$

\begin{table}[H]
%\centering
\begin{tabular}{|c|c|c|c|}
\hline
$N_i$ & $n_1$ & $n_2$ & $n_3$ \\
\hline
$0$ & $0$ & $0$ & $0$\\
$1$ & $1$ & $1$ & $0$\\
$2$ & $0$ & $1$ & $1$\\
$3$ & $1$ & $0$ & $1$\\
$4$ & $2$ & $0$ & $0$\\
$5$ & $0$ & $2$ & $0$\\
$6$ & $0$ & $0$ & $2$\\
\hline
\end{tabular}

\caption{Como vemos cada fila $n_1$,$n_2$ y $n_3$ debe sumar $N_{rad}$ excepto la primera fila y 
para un $N_{rad}=2$ tenemos $N_{coef}=7$.}

\label{tab:res}
\end{table}

En nuestro caso no utilizaremos el ajuste con el tensor de difusi\'on en cada punto, es decir 
no rotamos los $q$ y cada $U_i$ toma el mismo valor en cada punto y el mismo se corresponde con la 
media de difusividad la cual se define en el c\'odigo de manera razonable.
% cito rutger: if you set dti_scale_estimation=False, that just means that ``mu'' is the same 
% everywhere and corresponds to the mean diffusivity, which the code presets to something
% reasonable.











%%%%%%%%%%%%%%%%%%%%%%%%%%%%%%%%%%%%%%%%%%%%%%%%%%%%%
%% HASTA ACA ARTE %%%%%%%%%%%%%%%%%%%%%%%%%%%%%%%%%%%
%%%%%%%%%%%%%%%%%%%%%%%%%%%%%%%%%%%%%%%%%%%%%%%%%%%%%
%%%%%%%%%%%%%%%%%%%%%%%%%%%%%%%%%%%%%%%%%%%%%%%%%%%%%
%%%%%%%%%%%%%%%%%%%%%%%%%%%%%%%%%%%%%%%%%%%%%%%%%%%%%






% TODO
%[MRI zoom 0]
%spins+frecuencia de larmor+gradientes+resonancia $\rightarrow$ voltios en una antena receptora
%se\~nal recibida+fft $\rightarrow$ imagen 

%[dMRI zoom 0]
%e(q) = s(q)/S0 $\rightarrow$ lo unico que pudo haber cambiado es por culpa de la difusion


%[dMRI bueno para + enfermedades]
% Diffusion MRI (dMRI) is therefore the method of choice to probe microstructure, because it is 
%sensitive to the micron-scale displacement of water molecules, and is therefore strongly affected 
%by 
%the number, orientation and permeability of barriers (e.g. myelin) and the presence of various 
%cell 
%types and organelles (e.g. neurons, dendrites, axons, neurofilaments and microtubules) in living 
%tissue (Beaulieu, 2002). In part\'icular, dMRI can detect microstructural changes in the white 
%matter 
%related to myelination and demyelination, pruning, axonal loss, and has, for this reason, become 
%particularly useful for assessing damage in white matter pathologies (Horsfield and Jones, 2002).




\chapter{La guerra de las galaxias}
\section{Infancia y juventud}
{\begin{small}%
\begin{flushright}%
\it
There's nothing for me now.
I want to learn the ways of\\ the Force and become a Jedi like my father. \\
--Luke Skywalker
\end{flushright}%
\end{small}%
\vspace{.5cm}}

\section{Rescate de la princesa}
{\begin{small}%
\begin{flushright}%
\it
Here's where the fun begins!\\
--Han Solo
\end{flushright}%
\end{small}%
\vspace{.5cm}}

\section{Sacrificio y victoria}
{\begin{small}%
\begin{flushright}%
\it
This will be a day long remembered.\\ It has seen the end of Kenobi.\\ It will soon see the end of the Rebellion.\\
--Darth Vader
\end{flushright}%
\end{small}%
\vspace{.5cm}}

%% ...
\chapter{El imperio contraataca}
\chapter{El regreso del Jedi}

%%%% BIBLIOGRAFIA
\backmatter
\bibliographystyle{plainnat}
%\bibliography{/home/routeatlas/Documentos/cscomputacion/inria/tesis/latex/references.bib}
%\bibliography{/user/lgomez/home/Documents/tesis/latex/references2.bib}
\bibliography{references.bib}


\end{document}
