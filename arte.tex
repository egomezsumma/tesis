
\chapter{Estado del Arte}

En MRI la resoluci\'on t\'ipica de las im\'agenes es del orden de los mil\'imetros, mientras las 
escalas de las longitudes en los tejidos neuronales son del orden de los micrones. La t\'ipica 
resoluci\'on de los voxeles de hoy en d\'ia es de $2\times2\times2 mm^3$. Esta diferencia en 
escala, provoca groseros errores en los resultados de las tractografias. Es por ello, que es de 
suma importancia mejorar la resoluci\'on de las im\'agenes de disfusi\'on. Reducir el volumen del 
voxel es desafiante porque trae aparejado una mayor relaci\'on se\~nal ruido en las mediciones. 

Debido a la escacez de escáneres con buena resolución y la imposibilidad de aplicarlos en 
escenarios 
clínicos, recientemente varias técnicas han sido propuestas para obtener imágenes con alta 
resolución. Como 
por ejemplo transferencia de calidad de imágenes (\textit{image quality transfer} en ingles). La 
cual 
consiste en explotar la 
información presente en imágenes adquiridas con alta resolución para mejorar otras con baja 
resolución 
\citep{Alexander2014}. También la técnica conocida como súper-resolución, en 
la cual se obtiene una imagen con alta resolución utilizando varias imágenes en baja resolución. 
Estas se adquieren siguiendo un esquema determinado y deben ser del mismo sujeto 
\citep{Irani1993,Robinson2010,Ning2016}. Estas técnicas pueden mejorar la resolución de las 
imágenes 
usando la 
representación de la señal de difusión, tal como la adquieren los escáners, o pueden utilizar algún 
modelo de la señal 
de difusión, de los tantos que existen en el área.
%\subsection{Otras propuestas con el mismo objetivo}
Por ejemplo en el trabajo de \citet{Alexander2014} proponen hacer transferencia de 
calidad de imágenes utilizando algoritmos de aprendizaje automático (\textit{machine learning} del 
ingles) usando los 
modelos DTI y NODDI \citep{Zhang2012}. 
%Entrenando a dicho algoritmo con un conjunto de datos de alta calidad. 
La novedad de este trabajo es que el realce se hace sobre los parámetros del modelo de difusión y 
no 
sobre la 
señal de difusión tal como la adquieren los escáners. En cambio el trabajo de \citet{Ning2016} 
combinan la t\'ecnica 
de súper-resolución con el modelo de difusión \textit{compressed-sensing} \citep{Naidoo2015}. A 
diferencia del primer 
trabajo, utilizan un algoritmo de optimización convexa para reconstruir la imagen con alta calidad. 
En ambos casos el 
algoritmo depende fuertemente del modelo seleccionado para representar la señal de difusión.
