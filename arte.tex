\chapter{Estado del Arte}

\begin{verbatim}
IQT, Super resolucion Trabajo de Alexamnder y Daniel 
\end{verbatim}


\section{IQT}
[problema de la resoluci\'on]
En MRI la resoluci\'on t\'ipica de las im\'agenes es del orden de los mil\'imetros, mientras las 
escalas de las longitudes en los tejidos neuronales son del orden de los micrones. La t\'ipica 
resoluci\'on de los voxeles de hoy en d\'ia es de $2\times2\times2 mm^3$. Esta diferencia en 
escala, provoca groseros errores en los resultados de las tractografias. Es por ello, que es de 
suma importancia mejorar la resoluci\'on de las im\'agenes de disfusi\'on. Reducir el volumen del 
voxel es desafiante porque trae aparejado una mayor relaci\'on se\~nal ruido en las mediciones. 

Para apalear este problema se han propuestos varios m\'etodos para obtener im\'agenes de alta 
resoluci\'on. Los cuales caen en dos grandes categor\'ias. El primero de ellos toman como 
entrada una sola imagen en baja resoluci\'on y por medio de t\'ecnicas de 
\textit{interpolaci\'on} obtienen una imagen de mayor resoluci\'on. El segundo grupo, m\'as 
conocido como \textit{super-resolution}, toma como entrada varias im\'agenes en baja resoluci\'on y 
con ellas intenta construir una en alta resoluci\'on. [falta]
