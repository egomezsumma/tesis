\documentclass[a4paper,10pt]{article}%
%\documentclass[a4paper,10pt]{scrartcl}

\usepackage{float}
\usepackage{graphicx}
\usepackage[utf8]{inputenc}
\usepackage[english]{babel}
\usepackage[left=3cm,right=3cm,bottom=3.5cm,top=3.5cm]{geometry}
\usepackage{natbib}
\usepackage[utf8]{inputenc}


%opening
\title{}
\author{Leonel Exequiel Gómez}


\usepackage{graphicx}
\graphicspath{ {img/} }

%Path in Unix-like (Linux, OsX) format
\graphicspath{ {img/} }



\begin{document}

\maketitle



\section{Introducción}
Implementaremos un experimento basado en el paper de \citet{Alexander2014} y \citet{Ning2016}. 
El objetivo del experimento es dado un conjunto de imágenes de difusión de 
alta calidad, usar la información presente en las mismas para mejorar la 
resolución de imágenes de difusión de menor calidad. Esto es conocido como 
transferencia de calidad de imágenes (o \textit{image quality transfer} del 
término en ingles). 

Usaremos el conjunto de datos \textit{standford hardi}, provisto por la 
librer\'ia \textit{Dipy} como nuestra imagen de alta calidad. El mismo tiene 
una resolución espacial de $2\times2\times2\ mm^3$ con 160 direcciones 
gradientes con $b=2000\ s/mm^2$. Artificialmente generaremos (utilizando el 
m\'etodo \textit{reslice} de Dipy que hace una interpolación cubica) una imagen 
de difusión de baja calidad, a partir de la imagen provista por \textit{standford 
hardi}. Luego mejoraremos la resolución de la misma y la 
compararemos con la original (figura \ref{inputimg}).


\begin{figure}[h]
%\centering
\includegraphics[width=0.5\textwidth]{hardi.pdf}
\includegraphics[width=0.5\textwidth]{downsampling.pdf}
\caption{Corte coronal del conjunto de datos $S_0$. A la izquierda la imagen 
original de $81\times106\times76$ y a la derecha su equivalente disminuida en 
resolución de $43\times55\times40$. El rectángulo muestra el sub volumen con el 
que trabajaremos} 
\label{inputimg}
\end{figure}%


%Utilizaremos un subvolumen de $12\times12\times12$ con 6 valores de gradientes 
%diferentes, todos ellos con $b=2000$.

\section{Experimento}

Dada una imagen en baja resolución predeciremos su equivalente en alta 
resolución. Para ello denotaremos como $Y^{LR}$ a la imagen en baja resolución 
representada como un vector columna de dimensión $N^{vlrb}$ con $N^{vlrb}$ la 
cantidad de voxeles por la cantidad de gradientes de la imagen. Denotaremos 
como $Y^{HR}$ a la imagen en alta resolución representada como un vector 
columna de dimensión $N^{vhrb}$ con $N^{vhrb}$ la cantidad de voxeles por la 
cantidad de gradientes de la imagen. Al igual que \citep{Ning2016} consideramos 
a las imágenes en baja resolución como la versión sub muestreada de su 
equivalente en alta resolución. Luego el modelo de adquisición de la imagen de 
alta resolución lo podemos expresar como

$$  Y^{LR} = GY^{HR}$$

Donde $G$ es la matriz de sub muestreo de la resolución espacial. En 
nuestra implementación obtendremos $G$ entrenando un algoritmo de 
\textit{machine learning} con pares de sub volúmenes en baja resolución con su 
correspondiente en alta resolución. Para eso construiremos el conjunto de 
entrenamiento $T=\{x_i, y_i\}_{i}^{|T|}$, donde cada $x_i$ tiene dimensión $N^{vlrb}$ y los $y_i$ 
dimensión $N^{vhrb}$. En este caso lo construimos a partir de un 
mismo sub volumen pero con ruido $gauseano$ agregado artificialmente .


Usaremos el algoritmo de \textit{machine learning} \textit{LinearRegression},
provisto por 
la librer\'ia \textit{Scikit Learn}. Para entrenar el algoritmo dispusimos 10 
pares de sub vol\'umenes. Los tama\~nos de los sub 
vol\'umenes fueron de $6\times6\times6$ y $12\times12\times12$ para la baja y 
para la alta resolución respectivamente. Ambos con 6 gradientes diferentes 
(todos ellos con $b=2000\ s/mm^2$).



Con el algoritmo \textit{machine learning} computaremos la transformación 
lineal $G=YX^{\dagger}$ donde $Y$ tiene como columnas los $y_i$, $X$ tiene como 
columnas los $x_i$ y $X^{\dagger}$ es la pseudo inversa de $X$.
%con el conjunto de imágenes que contamos para entrenar
Luego de obtener $G$, planteamos el siguiente problema de optimización convexa 
para obtener la imagen en alta calidad buscada

$$  \min_{Y^{HR}} \{ || G Y^{HR} - Y^{LR} ||^2  \}$$

Dicha optimización la calculamos usando la librería \textit{CVXPY}.

Probamos el algoritmo para diferentes conjuntos de entrenamiento. Para 
ello, agregamos aleatoriamente ruido $gauseano$ a cada par del conjunto de 
entrenamiento. Es decir, para cada par en el conjunto de entrenamiento 
$T^{SNR}$ agregamos ruido con una relación señal ruido igual a $SNR$ (lo 
hicimos con $SNR$\footnote{SNR lo definimos aquí como $S_0/\sigma $, 
donde $\sigma$ es el desvío estándar de la distribución gauseana.}$= 25,\ 50,\ 
75,\ 100$ y $Inf$). A partir de estos conjuntos obtuvimos distintas $G^{SNR}$. 
En la figura \ref{snr_lr_hr} podemos ver los sub volumen en alta y baja 
resolución originales y uno de los pares usados para entrenar al azar. 

\begin{figure}[H]
\centering
\includegraphics[width=0.5\textwidth]{hr_Tsnr75.pdf}
\includegraphics[width=0.5\textwidth]{lr_Tsnr75.pdf}
\caption{En la fila arriba de izquierda a derecha la imagen en alta 
resolución (original) y la equivalente con el ruido adicionado respectivamente. 
En la fila de abajo, de izquierda a derecha la imagen en baja resolución 
sin ruido y la equivalente con el ruido adicionado 
respectivamente. El par con ruido pertenece al conjunto de entrenamiento 
$T^{75}$} 
\label{snr_lr_hr}
\end{figure}%

En la sección de resultados compararemos la imagen original con la reconstruida por el 
experimento. Luego compararemos los resultados según el ruido agregado 
artificialmente al conjunto 
de entrenamiento con el que estimamos $G$. De esta manera veremos como se comporta nuestro 
m\'etodo cuando el conjunto de datos con el que contamos para hacer la transferencia de calidad no 
es tan bueno.

\section{Resultados}

Comparamos la imagen original con la reconstruida usando las siguientes 
métricas:

\begin{itemize}
% \item Error cuadrático medio por parámetro DTI: Tomamos la imagen en su 
%representación DTI y calculamos el error cuadrático medio de todos los voxeles 
%del volumen por cada uno de los seis parámetros del modelo DTI.
 \item Error cuadrático medio por voxel: Tomamos la imagen en su 
representación DWI y calculamos el error cuadrático medio ($ECM$) por cada voxel del 
volumen.
%$$\frac{1}{160} \sum_{i=1}^{160} (S_{reconstruida}(q_i)-S_{original}(q_i))^2 $$ 
 \item Error cuadrático medio por gradiente: Tomamos la imagen en su 
representación DWI y calculamos el error cuadrático medio de todos los voxeles 
por cada gradiente. Esta métrica nos permite observar cuanto dista la imagen 
reconstruida con la original según el gradiente.
 \item Coeficiente de determinación: El Coeficiente de Determinación, es el cuadrado del 
coeficiente de correlación de Pearson. Nos da la proporción de variación de la variable a predecir. 
Si la proporción es igual a 0, quiere decir que el modelo no es bueno. Si esta cerca de uno 
quiere decir que el modelo es bueno y para valores negativos significa que nuestro conjunto de 
datos no se comporta de manera lineal.
 %\item Histograma: Calculamos el histograma de una rebanada coronal para un 
%b-valor nulo y uno no nulo. Ver el histograma de una imagen nos ayuda analizar 
%sobre el contrastarte de la misma.
\end{itemize}





En la tabla \ref{tab:res} podemos ver el valor mínimo alcanzado por el algoritmo de minimización, el 
tiempo de ejecución del mismo, y las normas de la imagen estimada y la diferencia entre la estimada 
y la original (siendo $||Y^{HR}_{original}|| = 18093$), seg\'un la se\~nal ruido agregada al 
conjunto de entrenamiento. Adem\'as por cada conjunto de entrenamento probamos el algoritmo de 
minimizaci\'on con distintos sub volumenes $(Y^{LR})^{SNR}$ con distintos valores se\~nal ruido.




\begin{table}[H]
\centering
\begin{tabular}{ |c|c|c|c|c|c| }
 \hline
 SNR & Valor Óptimo & Tiempo  & $||Y^{HR}||$ & $||Y^{LR}||$ & $||Y^{HR}-Y^{HR}_{original}||$\\
 \hline 
 inf & $3034\pm494$  & $5'07''$  & $17908\pm16$ & $6427\pm5$  & $859\pm3$\\
 200 & $3058\pm530$  & $5'02''$  & $17926\pm20$ & $6433\pm15$ & $811\pm3$  \\
 100 & $3054\pm556$  & $5'08''$  & $17917\pm3$  & $6430\pm9$  & $723\pm1$  \\  
 75  & $3063\pm540$  & $5'33''$  & $17948\pm38$ & $6441\pm21$ & $653\pm6$  \\
 50  & $3083\pm559$  & $5'04''$  & $17893\pm46$ & $6420\pm9$  & $713\pm14$ \\  
 25  & $3051\pm500$  & $5'09''$  & $17931\pm18$ & $6435\pm13$ & $770\pm3$  \\  
 %00  & $38653.04$ & $0'0''$    & $108.34$ & $0$  & $16497.96$  \\
\hline
\end{tabular}
\caption{El valor alcanzado por el algoritmo de minimización, el tiempo de ejecución, la 
norma de la imagen estimada y la norma de la diferencia entre la estima y la original, en cada 
caso según el valor de se\~nal ruido agregado al conjunto de entrenamiento. El algoritmo fue 
probado con disitintas $Y^{LR}$ de entradas. Todas ellas con un valor diferente de se\~nal ruido.}
\label{tab:res}
\end{table}

La figura \ref{mse_cmp} muestra una rebanada coronal de la imagen de difusión original, la producida 
por el m\'etodo y el error cuadrático medio entre las \'ultimas dos. Como 
podemos apreciar solo algunos voxeles toman valores relativamente altos en la 
imagen $ECM$.

\begin{figure}[H]
\centering        
\includegraphics[scale=0.7]{mse_snr50_voxels_b1_cmp.pdf}
\caption{De izquierda a derecha una rebanada coronal de un gradiente arbitrario de la imagen 
original, la producida por el m\'etodo y el error cuadrático medio por voxel de la 
representación DWI de la imagen. La imagen fue resultado del m\'etodo con $T^{75}$ y 
$(Y^{LR})^{75}$.}
\label{mse_cmp}
\end{figure}


La figura \ref{cdet_cmp} muestra también una rebanada coronal de la imagen 
original, la producida por el m\'etodo y el coeficiente de determinación entre las \'ultimas dos. 
Dicho coeficiente nos aporta información de cuan bueno fue el modelo que utilizamos para fitear 
los datos. En nuestro caso el modelo usado fue lineal y gran parte de los voxeles presentan un 
coeficiente de determinación por encima del $95\%$. Esto significa que el modelo elegido es bueno 
para este conjunto de entrada.

\begin{figure}[H]
\centering        
\includegraphics[scale=0.7]{cdet_snr50_voxels_b1_cmp.pdf}
\caption{De izquierda a derecha una rebanada coronal de un gradiente arbitrario de la imagen 
original, la producida por el m\'etodo y el el coeficiente de determinación por voxel de la 
representación DWI de la imagen. La imagen fue resultado del m\'etodo con $T^{75}$ y 
$(Y^{LR})^{75}$.}
\label{cdet_cmp}
\end{figure}


En la figura \ref{mse_por_gradientes_by_snr_todos} graficamos el error 
cuadrático medio de todo el 
sub volumen por cada uno de los 6 gradientes utilizados para los distintos conjuntos de 
entrenamiento. Como se puede ver el error se hace marcadamente mas grande cuando la relacion 
se\~nal ruido de la imagen disminuye. 



\begin{figure}[H]
\centering        
\includegraphics[scale=0.7]{mse_por_gradientes_by_snr_todos.pdf}
\caption{El error cuadrático medio por cada una de las 6 direcciones gradientes 
utilizadas por cada conjunto de entrenamiento con SNR igual a $25,\ 50,\ 75,\ 100$ y $200$. }
\label{mse_por_gradientes_by_snr_todos}
\end{figure}

\clearpage
\section{Conclusiones}
%
En el caso de los b-valores altos el error es m\'as abrupto en las regiones donde predomina el 
tejido isotrópico (como la materia gris, csf, etc).




El modelo propuesto se comporta bien para valores no tan altos de relación señal ruido. Como se ve 
en la figura \ref{mse_por_gradientes_by_snr_todos} cuando la relación señal ruido es 
inferior a 50 el error del método crece abruptamente. 

A simple vista (incluso los resultados con $SNR=25$) el modelo aproxima bien la 
imagen en cuanto a graficar la señal de difusión. Es decir, que respeta 
bastante bien el contraste entre voxeles. Más allá de que las magnitudes de los 
mismos difieran con las de la original y eso cause diferencias sensibles al 
calcular el $ECM$. Habría que probar estos resultados al intentar 
usar los datos para hacer cosas mas complejas como por ejemplo un tractograma.

%\clearpage
%\section{Ap\'endice}
%

Sean:

 $ vhr \in  Z_{ \geq 0} $
 
 $ bval \in  Z_{ \geq 0} $
 
 $Y^{hr} \in \Re^{vhr\times bval}$
 
 $Y^{lr} \in \Re^{vlr\times bval}$

 $G \in \Re^{vlr\times vhr}$


$$  \min_{Y^{hr}} \{ || G Y^{hr} - Y^{lr} ||^2  \}$$

Llamo a :

$f : \Re^{vhr\times bval} \rightarrow \Re^{vlr\times bval}$

$$f(Y^{hr}) = G Y^{hr} - Y^{lr} $$

$g:\Re^{vhr\times bval} \rightarrow \Re$

$$g(Y^{hr}) = ||f(Y^{hr})||^2 $$

con 

$f_{ij}(Y^{hr}) : \Re$

$$f_{ij}(Y^{hr}) = (\sum_{k=1}^{vhr} G_{ik} Y_{kj}^{hr}) - Y_{ij}^{lr} $$

Luego

$$g(Y^{hr}) = [ \sqrt{ \sum_{i=1}^{vlr} \sum_{j=1}^{bval} f_{ij}^2 (Y^{hr})} ]^2 $$


$$g(Y^{hr}) =  \sum_{i=1}^{vlr} \sum_{j=1}^{bval} f_{ij}^2 (Y^{hr}) $$


Para minimizar $g(Y^{hr})$ preciso $\nabla g \in \Re^{vhr\times bval} $.
Si nos enfocamos en un componente del gradiente, osea la derivada parcial con respecto a 
$Y_{ij}^{hr}$, al ser una sumatoria solo afecta a la derivada los sumandos que tiene a 
$Y_{ij}^{hr}$ es decir:

$$\frac{\partial g (Y^{hr})}{\partial Y_{ij}^{hr}} = \sum_{k=1}^{vlr} \frac{\partial 
f_{kj}^2(Y^{hr})}{\partial Y_{ij}^{hr}}$$



Por la derivada de la composición $(g \circ h)' (Y^{hr}) = g'(h(Y^{hr})) \ 
h'(Y^{hr})$ tenemos que 

$$\frac{\partial g (Y^{hr})}{\partial Y_{ij}^{hr}} = \sum_{k=1}^{vlr} 2 f_{kj}(Y^{hr}) 
\frac{\partial 
f_{kj}(Y^{hr})}{\partial Y_{ij}^{hr}}$$

y 


$$\frac{\partial f_{kj}(Y^{hr})}{\partial Y_{ij}^{hr}} = 0 + 0 + \cdots + 
\underbrace{(G_{ki}Y_{ij}^{hr})'}_{k=i} + 0 + \cdots $$

$$\frac{\partial f_{kj}(Y^{hr})}{\partial Y_{ij}^{hr}} = G_{ki}$$

luego



$$\frac{\partial g (Y^{hr})}{\partial Y_{ij}^{hr}} = \sum_{k=1}^{vlr} 2 f_{kj}(Y^{hr})G_{ki}$$


$$\frac{\partial g (Y^{hr})}{\partial Y_{ij}^{hr}} = \sum_{k=1}^{vlr} 2 [(\sum_{p=1}^{vhr} G_{kp} 
Y_{pj}^{hr}) - Y_{kj}^{lr}] G_{ki}$$



$$\frac{\partial g (Y^{hr})}{\partial Y_{ij}^{hr}} = 2 \sum_{k=1}^{vlr} G_{ki} (\sum_{p=1}^{vhr} 
G_{kp} 
Y_{pj}^{hr}) - Y_{kj}^{lr}$$



que equivale a su forma matricial


$$\nabla g (Y^{hr}) = 2 G^t(G Y^{hr}-Y^{lr})$$





\clearpage
\bibliographystyle{plainnat}
\bibliography{references.bib}



\end{document}
