\documentclass[a4paper,10pt]{article}
\usepackage[utf8]{inputenc}
\usepackage{setspace}


%\usepackage{float}
%\usepackage{graphicx}

%\usepackage[spanish]{babel}
\usepackage[left=3cm,right=3cm,bottom=3.5cm,top=3.5cm]{geometry}
\usepackage{natbib}

\title{El titulo}

\newcommand\textline[3][t]{%
  \par\smallskip\noindent\parbox[#1]{.658\textwidth}
  {\raggedright#2}%
  %\parbox[#1]{.333\textwidth}{\centering#3}%
  \parbox[#1]{.333\textwidth}{\raggedleft#3}%\par\smallskip%
}

%opening
%\author{Gómez, Leonel Exequiel}

\begin{document}

{\setstretch{0.5}
\textline[t]{\textbf{Director}}{\textbf{Alumno}}
\textline[t]{Demián Wassermann}{Leonel Exequiel G\'omez}
\textline[t]{Investigador Permanente Athena Project Team, INRIA}{LU: 436/07, UBA}
\textline[t]{demian.wassermann@inria.fr}{leexgo1987@gmail.com}
}

%\maketitle

\section*{Resumen}

\subsection{Necesidad que da origen a nuestro objetivo en el área}
La Resonancia Magnética de difusión (dMRI en ingles) es una técnica útil para generar imágenes de 
órganos internos del cuerpo, sin la necesidad de someter al mismo a un procedimiento quirúrgico. Las imágenes 
construidas por esta técnica están fuertemente relacionadas con la difusión de las partículas de agua en los tejidos. 
Estas proveen valiosa información sobre la micro-estructura de los diferentes tipos de tejidos del cerebro 
(LeBihan and breton 1985). En la actualidad, es la única técnica capaz de evaluar 
%de forma no invasiva 
la conectividad cerebral en un cerebro humano vivo y por lo tanto posee un enorme potencial para resolver problemas 
actuales en el área de la neurociencia. Los estándares de resolución de hoy en día de dichas imágenes, permiten hacer 
diversos análisis sobre conjuntos de fibras de mediano y gran tamaño. Sin embargo, estos análisis 
pierden precisión en regiones donde las fibras son muy pequeñas en comparación al tamaño del voxel (píxel en tres 
dimensiones). Consecuentemente, existe la necesidad de incrementar tanto la resolución espacial, angular y temporal de 
las imágenes dMRI. Dado que el dominio de muestreo de la señal que da origen a este tipo de imágenes, no es solo 
espacial. El mismo comprende además la intensidad del gradiente de difusión aplicado, la dirección del mismo y el 
tiempo de difusión. Es por ello que decimos que estas imágenes poseen resolución angular y temporal. La baja 
relaci\'on se\~nal ruido y el tiempo de adquisici\'on de las imágenes dMRI son unos de los principales factores que 
limitan la 
resolución espacial. %Los limites para obtener imágenes con alta resolución están relacionados con 
%el tiempo de adquisición de las mismas y la relación señal ruido (la cual decrece 
%con el tamaño voxel). 
Hoy en día, existen escáners que pueden 
producir imágenes con muy alta resolución. Pero los mismos son muy costosos e 
imprácticos para escenarios clínicos.

\subsection{Explicar técnicas auxiliares a utilizar}
Para obtener detalles mas finos sobre las características micro-estructurales a partir de las imágenes de difusión, se 
puede ajustar un modelo matemático a los datos adquiridos. Uno de los principales objetivos de los modelos de difusión 
es recuperar información más detallada sobre la orientación de las fibras, haciendo uso del 
conocimiento a priori que se posee sobre la anatomía de los tejidos neuronales. Entre ellos se 
encuentra el modelo \textit{Diffusion Tensor Imaging} (DTI). En el cual se representa la 
se\~nal de difusión con un tensor de segundo orden \citep{Basser1994}. También existen modelos mas 
complejos como por ejemplo MAPL \citep{Fick2016365}. El mismo pertenece a una familia de modelos que parte del enfoque 
de representar la señal de difusión con herramientas de análisis harmónico.
%una señal como una sumatoria de otras 
%señales mas básicas. 
%Con dicho modelo se propone extrapolar los valores de la se\~nal de difusión para aquellos valores del dominio en los 
%cuales la relaci\'on se\~nal ruido comienza a gobernar la se\~nal. 
Este extiende el pre-existente modelo llamado \textit{Mean Apparent Propagator} (MAP) (Ozarslan et al, [2013b]), 
utilizando la norma del Laplaciano de la se\~nal de difusión, como regularizaci\'on en el ajuste de los coeficientes 
que MAP define. 


\subsection{Propuestas que dan origen a nuestro objetivo}
Recientemente varias técnicas han sido propuestas para obtener imágenes con alta resolución. Como 
por ejemplo 
transferencia de calidad de imágenes (\textit{image quality transfer} en ingles). La cual 
consiste en explotar la 
información presente en imágenes adquiridas con alta resolución para mejorar otras con baja resolución 
\citep{Alexander2014}. También la técnica conocida como súper-resolución, en 
la cual se obtiene una imagen con alta resolución utilizando varias imágenes con baja resolución. 
La adquisición de las imágenes con baja 
resolución deben seguir un esquema determinado y ser del mismo sujeto (\cite{Irani1993,Robinson2010}; Greenspan et al 
2002; Gholipour et al 2010). Y otras técnicas que por medio de interpolación 
intentan aumentar la resolución de una imagen con baja resolución (\cite{Manjon2010}). Estas técnicas pueden mejorar la 
resolución de las imágenes usando la representación de la señal de difusión tal como la adquieren los escáners. O bien 
pueden utilizar algún modelo de la señal de difusión, de los tantos que existen en el área.




\subsection{Otras propuestas con el mismo objetivo}
Por ejemplo en el trabajo de \citet{Alexander2014} se propone hacer transferencia de 
calidad de imágenes utilizando algoritmos de aprendizaje automático 
(\textit{machine learning} del ingles) para reconstruir la imagen con alta 
calidad. Entrenando a dicho algoritmo con un conjunto de datos de alta 
calidad. La novedad de este trabajo es que el realce no se hace directamente 
sobre la señal de difusión tal como la adquieren los escáners, sino sobre los 
parámetros de algún modelo de difusión. En particular lo hacen para los modelos DTI y NODDI 
\citep{Zhang2012}. 
%La ventaja de hacer el entrenamiento sobre el modelo en vez de la se\~nal es que se puede utilizar 
%para extrapolar conjuntos de datos que no cumplan con las precondiciones para un modelo 
%determinado. 
En cambio el trabajo de \citet{Ning2016} combina la 
t\'ecnica de súper-resolución con el modelo \textit{compressed-sensing} \citep{Naidoo2015} de la señal de difusión. A 
diferencia del primer trabajo, utiliza un algoritmo de 
optimización convexa para reconstruir la imagen con alta calidad.

\subsection{El objetivo}
El objetivo de este trabajo es analizar la aplicación de la técnica de 
transferencia de calidad, utilizando el modelo \textit{Multi-Spherical Diffusion MRI} que toma en cuenta además 
la dimensión temporal \citep{Fick}. Para ello plantearemos el objetivo en términos de un problema 
de optimización convexa que tengan en cuenta tanto el modelo de la se\~nal como la correlaci\'on 
que existe entre voxeles vecinos. Tanto la complejidad espacial, como temporal de las 
operaciones necesarias para trabajar con estos vol\'umenes de datos, representan un desafío 
computacional, en cuanto a desarrollar un algoritmo eficiente que lleve a cabo el realce buscado. 
Utilizaremos la base de datos \textit{Human Connectome Proyect} \citep{Barch2013}, la cual posee 
gran cantidad de imágenes dMRI estándares de resolución muy alta.



\clearpage
\bibliographystyle{plainnat}
\bibliography{references.bib}












\end{document}
\grid
