\documentclass[a4paper,10pt]{article}
\usepackage[utf8]{inputenc}



%\usepackage{float}
%\usepackage{graphicx}

\usepackage[spanish]{babel}
\usepackage[left=3cm,right=3cm,bottom=3.5cm,top=3.5cm]{geometry}
\usepackage{natbib}


%opening
\title{El titulo}
\author{Gómez, Leonel Exequiel}

\begin{document}

\maketitle

\section{Resumen}

\subsection{Necesidad que da origen a nuestro objetivo en el area}
Las Resonancia Magnética de difusión (dMRI del termino en ingles) 
es una técnica útil para generar imágenes de órganos internos del cuerpo, sin 
la necesidad de someter al mismo a un procedimiento quirúrgico. Estas 
proveen valiosa información sobre la micro estructura de los diferentes 
tipos de tejidos del cerebro (ref a algun paper fundacional de dmri). Los 
estándares de resolución de hoy en día permiten hacer diversos análisis sobre 
conjuntos de fibras de mediano y gran tamaño. Sin embargo, estos análisis 
pierden precisión en regiones donde las fibras son muy pequeñas en comparación 
al tamaño del voxel. Consecuentemente, existe la necesidad de 
incrementar la resolución espacial de las imágenes dMRI. Los limites para 
obtener imágenes en alta resolución están fuertemente relacionados con el 
tiempo de adquisición de las mismas y la relación señal ruido (la cual decrese 
cuanto mas pequeño es el voxel). Hoy en día, existen escáners que pueden 
producir imágenes con muy alta resolución. Pero los mismos son muy costos e 
impracticos para escenarios clínicos.


\subsection{Propuestas que dan origen a nuestro objetivo}
Recientemente varias métodos han sido propuestos para obtener imágenes con 
alta resolución. Como por ejemplo transferencia de calidad imágenes utilizada 
en (ref a paper de daniel C alexander) que explota la información presente en 
imágenes adquiridas con alta resolución para mejorar la resolución de otras con 
baja resolución. También la técnica conocida como súper-resolución, que 
consiste en obtener una imagen con alta resolución utilizando varias 
imágenes con baja resolución del mismo sujeto (Irani and peleg 1993; 
Greenspan et al 2002; Greenspan et al 2009; Gholipour et al 2010; cs 2015). Y 
otras que por medio de técnicas de interpolación intentan aumentar la resolución 
de una imagen con baja resolución (Manjon et al 2010).


\subsection{Explicar tecnicas auxiliares a utilizar}
Optimizacion convexa, Mapl

\subsection{Otras propuestas con el mismo objetivo}
Los dos papper de realce

\subsection{El objetivo}
Combinar la tecnica de trasn de calidad, utilizando el modelo MAPL que toma en 
cuenta ademas la dimension temporal. 





\clearpage
\bibliographystyle{plainnat}
\bibliography{references.bib}












\end{document}
