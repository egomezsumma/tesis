\documentclass[a4paper,10pt]{article}
\usepackage[utf8]{inputenc}



%\usepackage{float}
%\usepackage{graphicx}

\usepackage[spanish]{babel}
\usepackage[left=3cm,right=3cm,bottom=3.5cm,top=3.5cm]{geometry}
\usepackage{natbib}


%opening
\title{El titulo}
\author{Gómez, Leonel Exequiel}

\begin{document}

\maketitle

\section{Resumen}

\subsection{Necesidad que da origen a nuestro objetivo en el area}
Las Resonancia Magnética de difusión (dMRI del termino en ingles) 
es una técnica útil para generar imágenes de órganos internos del cuerpo, sin 
la necesidad de someter al mismo a un procedimiento quirúrgico. Estas 
proveen valiosa información sobre la micro estructura de los diferentes 
tipos de tejidos del cerebro (ref a algun paper fundacional de dmri). Los 
estándares de resolución de hoy en día permiten hacer diversos análisis sobre 
conjuntos de fibras de mediano y gran tamaño. Sin embargo, estos análisis 
pierden precisión en regiones donde las fibras son muy pequeñas en comparación 
al tamaño del voxel. Consecuentemente, existe la necesidad de 
incrementar la resolución espacial de las imágenes dMRI. Los limites para 
obtener imágenes en alta resolución están fuertemente relacionados con el 
tiempo de adquisición de las mismas y la relación señal ruido (la cual decrese 
cuanto mas pequeño es el voxel). Hoy en día, existen escáners que pueden 
producir imágenes con muy alta resolución. Pero los mismos son muy costos e 
impracticos para escenarios clínicos.


\subsection{Propuestas que dan origen a nuestro objetivo}
Recientemente varias métodos han sido propuestos para obtener imágenes con 
alta resolución. Como por ejemplo transferencia de calidad imágenes utilizada 
en (ref a paper de daniel C alexander) que explota la información presente en 
imágenes adquiridas con alta resolución para mejorar la resolución de otras con 
baja resolución. También la técnica conocida como súper-resolución, que 
consiste en obtener una imagen con alta resolución utilizando varias 
imágenes con baja resolución del mismo sujeto con un esquema de 
adquisición determinado (Irani and peleg 1993; Greenspan et al 2002; Greenspan 
et al 2009; Gholipour et al 2010; cs 2015). Y otras que por medio de técnicas 
de interpolación intentan aumentar la resolución de una imagen con baja 
resolución (Manjon et al 2010). Estas técnicas pueden mejorar la resolución de 
las imagenes usando la representacion dwi de la señal de disfusión (tal como la 
adquieren los escaners). O bien pueden utilizar algun modelo de la señal de 
difusion, de los tantos que existen en el area.


\subsection{Explicar técnicas auxiliares a utilizar}
Modelos de difusion, Mapl 

\subsection{Otras propuestas con el mismo objetivo}
Por ejemplo en el trabajo de (ref paper 1) se proponen hacer transferencia de 
calidad de imágenes utilizando algoritmos de aprendizaje automático 
(\textit{machine learning} del ingles) para reconstruir la imagen con alta 
calidad. Entrenando a dichos algoritmo con un conjuntos de datos de alta 
calidad. La novedad de este trabajo es que el realce no se hace directamente 
sobre la representación dwi de la imagen sino sobre los parametros de algun 
modelo de la señal de difusión. En particular lo hacen para los modelos DTI (ref 
dti) y NODDI (ref NODDI). La ventaja (bla bla NODDI no se puede sin multi 
shel). En cambio el trabajo de (paper 2) combina la tecnica de super-resolución 
con un modelo especifico de la señal de difusion (compressed-sensing). A 
diferencia del primero trabajo, utiliza un algoritmo de optimización convexa 
para reconstruir la imagen con alta calidad.

\subsection{El objetivo}
El objetivo de este trabajo es analizar la aplicación de la técnica de 
transferencia de calidad, utilizando el modelo MAPL que toma en cuenta además 
la dimensión temporal. Tanto la complejidad espacial, como temporal las 
operaciones necesarias, representan un desafío computacional, en cuanto a 
desarrollar un algoritmo eficiente que lleve a cabo el realce buscado.





\clearpage
\bibliographystyle{plainnat}
\bibliography{references.bib}












\end{document}
