\documentclass[a4paper,10pt]{article}
\usepackage[utf8]{inputenc}



%\usepackage{float}
%\usepackage{graphicx}

%\usepackage[spanish]{babel}
\usepackage[left=3cm,right=3cm,bottom=3.5cm,top=3.5cm]{geometry}
\usepackage{natbib}


%opening
\title{El titulo}
\author{Gómez, Leonel Exequiel}

\begin{document}

\maketitle

\section{Resumen}

\subsection{Necesidad que da origen a nuestro objetivo en el area}
La Resonancia Magnética de difusión (dMRI en ingles) es una técnica útil para generar imágenes de 
órganos internos del 
cuerpo, sin la necesidad de someter al mismo a un procedimiento quirúrgico. Las imágenes construidas por esta técnica 
están fuertemente relacionadas con la difusión de las partículas de agua en los tejidos. Estas proveen valiosa 
información sobre la micro estructura de los diferentes tipos de tejidos del cerebro (LeBihan and 
bretorn 1985). Los estándares de resolución de hoy en día de dichas imágenes, permiten hacer 
diversos análisis sobre conjuntos de fibras de mediano y gran tamaño. Sin embargo, estos análisis 
pierden precisión en regiones donde las fibras son muy pequeñas en comparación 
al tamaño del voxel (píxel en tres dimensiones). Consecuentemente, existe la necesidad de 
incrementar la resolución espacial de las imágenes dMRI. Los limites para 
obtener imágenes con alta resolución están fuertemente relacionados con el 
tiempo de adquisición de las mismas y la relación señal ruido (la cual decrese 
cuanto mas pequeño es el voxel). Hoy en día, existen escáners que pueden 
producir imágenes con muy alta resolución. Pero los mismos son muy costos e 
impracticos para escenarios clínicos.


\subsection{Propuestas que dan origen a nuestro objetivo}
Recientemente varias técnicas han sido propuestas para obtener imágenes con alta resolución. Como 
por ejemplo 
transferencia de calidad de imágenes (\textit{image quality transfer} en ingles). La cual 
consiste en explotar la 
información presente en imágenes adquiridas con alta resolución para mejorar otras con baja resolución 
\citep{Alexander2014}. También la técnica conocida como súper-resolución, en 
la cual se obtiene una imagen con alta resolución utilizando varias imágenes con baja resolución. 
La adquisición de las imágenes con baja 
resolución deben seguir un esquema determinado y ser del mismo sujeto (Irani and peleg 1993; Greenspan et al 
2002; Greenspan et al 2009; Gholipour et al 2010; cs 2015). Y otras que por medio de técnicas de interpolación intentan 
aumentar la resolución de una imagen con baja resolución (Manjon et al 2010). Estas técnicas pueden mejorar la 
resolución de las imágenes usando la representación de la señal de difusión tal como la adquieren los escaners. O bien 
pueden utilizar algún modelo de la señal de difusión, de los tantos que existen en el área.


\subsection{Explicar técnicas auxiliares a utilizar}
Los modelos de difusión tienen como objetivo recuperar información más detallada sobre la orientación de las fibras a 
partir de la señal de difusión. Entre ellos se encuentra el modelo \textit{Diffusion Tensor Imaging} (DTI). En el cual 
se representa la difusión de las partículas de agua con un tensor de segundo orden. También existen modelos mas 
complejos como por ejemplo MAPL \citep{Fick2016365}. Dicho modelo se propone extrapolar los 
valores de la se\~nal de difusion para aquellos valores del dominio para los cuales la relaci\'on 
se\~nal ruido comienza a gobernar la se\~nal. Este extiende el pre-existente modelo MAP (ref 
map), utilizando la norma del Laplaciano de la se\~nal de difusion, como regularizaci\'on en el 
fiteo de los coeficientes que MAP define.


\subsection{Otras propuestas con el mismo objetivo}
Por ejemplo en el trabajo de \citet{Alexander2014} se proponen hacer transferencia de 
calidad de imágenes utilizando algoritmos de aprendizaje automático 
(\textit{machine learning} del ingles) para reconstruir la imagen con alta 
calidad. Entrenando a dichos algoritmo con un conjuntos de datos de alta 
calidad. La novedad de este trabajo es que el realce no se hace directamente 
sobre la representación dwi de la imagen sino sobre los parametros de algun 
modelo de la señal de difusión. En particular lo hacen para los modelos DTI (ref 
dti) y NODDI (ref NODDI). La ventaja (bla bla NODDI no se puede sin multi 
shel). En cambio el trabajo de \citet{Ning2016} combina la tecnica de super-resolución 
con un modelo especifico de la señal de difusion (compressed-sensing). A 
diferencia del primero trabajo, utiliza un algoritmo de optimización convexa 
para reconstruir la imagen con alta calidad.

\subsection{El objetivo}
El objetivo de este trabajo es analizar la aplicación de la técnica de 
transferencia de calidad, utilizando el modelo MAPL que toma en cuenta además 
la dimensión temporal. Tanto la complejidad espacial, como temporal de las 
operaciones necesarias, representan un desafío computacional, en cuanto a 
desarrollar un algoritmo eficiente que lleve a cabo el realce buscado.





\clearpage
\bibliographystyle{plainnat}
\bibliography{references.bib}












\end{document}
