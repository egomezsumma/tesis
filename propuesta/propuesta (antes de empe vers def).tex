\documentclass[a4paper,10pt]{article}
\usepackage[utf8]{inputenc}
\usepackage{setspace}


%\usepackage{float}
%\usepackage{graphicx}

%\usepackage[spanish]{babel}
\usepackage[left=3cm,right=3cm,bottom=3.5cm,top=3.5cm]{geometry}
\usepackage{natbib}

\title{El titulo}

\newcommand\textline[3][t]{%
  \par\smallskip\noindent\parbox[#1]{.333\textwidth}
  {\raggedright#3}%
  %\parbox[#1]{.333\textwidth}{\centering#3}%
  \parbox[#1]{.658\textwidth}{\raggedleft#2}%\par\smallskip%
}

%opening
%\author{Gómez, Leonel Exequiel}

\begin{document}

{\setstretch{0.5}
\textline[t]{\textbf{Director}}{\textbf{Alumno}}
\textline[t]{Demián Wassermann}{Leonel Exequiel G\'omez}
\textline[t]{Investigador Permanente Athena Project Team, INRIA}{LU: 436/07, UBA}
\textline[t]{demian.wassermann@inria.fr}{leexgo1987@gmail.com}
}

%\maketitle

%\section*{Resumen}



\vspace{2cm}


%\subsection{Necesidad que da origen a nuestro objetivo en el área}
La Resonancia Magnética de difusión (dMRI en ingles) es una técnica útil para generar imágenes del 
cerebro, sin la necesidad de someter al paciente a un procedimiento quirúrgico. Las imágenes 
construidas por esta técnica están fuertemente relacionadas con la difusión de las partículas de agua en los tejidos. 
Estas proveen valiosa información sobre la micro-estructura de los diferentes tipos de tejidos del cerebro 
(LeBihan and breton 1985). En la actualidad, es la única técnica capaz de evaluar 
%de forma no invasiva 
la conectividad neuronal en un cerebro humano vivo y por lo tanto posee un enorme potencial para resolver problemas 
actuales en el área de la neurociencia. Los estándares de resolución de hoy en día de dichas imágenes, permiten hacer 
diversos análisis sobre conjuntos de fibras de mediano y gran tamaño. Sin embargo, estos análisis 
pierden precisión en regiones donde los grupos de axones son muy pequeñas en comparación al tamaño del voxel (4 ordenes 
de magnitud más pequeños). Consecuentemente, existe la necesidad de incrementar tanto la resolución 
espacial, angular y temporal de 
las imágenes dMRI. Dado que el dominio de muestreo de la señal que da origen a este tipo de imágenes, no es solo 
espacial. Este comprende además la intensidad del gradiente de difusión aplicado, la dirección del mismo y el 
tiempo de difusión (espacio $q$ \citep{CALLAGHAN1990177}). Es por ello que decimos que estas imágenes poseen resolución 
angular y temporal. La baja relación señal ruido y el tiempo de adquisición de las imágenes dMRI son unos de los 
principales factores que limitan la resolución de las mismas. %Los limites para obtener imágenes con alta resolución 
%están relacionados con 
%el tiempo de adquisición de las mismas y la relación señal ruido (la cual decrece 
%con el tamaño voxel). 
Hoy en día, existen escáners que pueden 
producir imágenes con muy alta resolución. Pero los mismos son muy costosos e 
imprácticos para escenarios clínicos.


%\subsection{Explicar técnicas auxiliares a utilizar}
Para obtener detalles más finos sobre las características micro-estructurales del tejido cerebral, como por ejemplo el 
diámetro axonal promedio en voxel, se puede ajustar un modelo matemático a los datos adquiridos. Uno de los principales 
objetivos de los modelos de difusión es recuperar información más detallada sobre la orientación de las fibras, haciendo 
uso del conocimiento a priori que se posee sobre la anatomía de los tejidos neuronales. Entre ellos se 
encuentra el modelo \textit{Diffusion Tensor Imaging} (DTI). En el cual se representa la 
se\~nal de difusión con un tensor de segundo orden \citep{Basser1994}. También existen modelos más 
complejos como por ejemplo MAPL \citep{Fick2016365}. El mismo pertenece a una familia de modelos que parte del enfoque 
de representar la señal de difusión con herramientas de análisis harmónico.
%una señal como una sumatoria de otras 
%señales mas básicas. 
%Con dicho modelo se propone extrapolar los valores de la se\~nal de difusión para aquellos valores del dominio en los 
%cuales la relaci\'on se\~nal ruido comienza a gobernar la se\~nal. 
Este extiende el pre-existente modelo llamado \textit{Mean Apparent Propagator} (MAP) (Ozarslan et al, [2013b]), 
utilizando la norma del Laplaciano de la se\~nal de difusión, como regularizaci\'on en el ajuste de los coeficientes 
que MAP define. 


%\subsection{Propuestas que dan origen a nuestro objetivo}

Como consecuencia a los pocos escáners con buena resolución y la imposibilidad de aplicarlos en escenarios 
clínicos, recientemente varias técnicas han sido propuestas para obtener imágenes con alta resolución. Como 
por ejemplo transferencia de calidad de imágenes (\textit{image quality transfer} en ingles). La cual 
consiste en explotar la 
información presente en imágenes adquiridas con alta resolución para mejorar otras con baja resolución 
\citep{Alexander2014}. También la técnica conocida como súper-resolución, en 
la cual se obtiene una imagen con alta resolución utilizando varias imágenes en baja resolución. 
Estas se adquieren siguiendo un esquema determinado y deben ser del mismo sujeto 
\citep{Irani1993,Robinson2010,Ning2016}. Estas técnicas pueden mejorar la resolución de las imágenes usando la 
representación de la señal de difusión, tal como la adquieren los escáners, o pueden utilizar algún modelo de la señal 
de difusión, de los tantos que existen en el área.
%\subsection{Otras propuestas con el mismo objetivo}
Por ejemplo en el trabajo de \citet{Alexander2014} proponen hacer transferencia de 
calidad de imágenes utilizando algoritmos de aprendizaje automático (\textit{machine learning} del ingles) usando los 
modelos DTI y NODDI \citep{Zhang2012}. 
%Entrenando a dicho algoritmo con un conjunto de datos de alta calidad. 
La novedad de este trabajo es que el realce se hace sobre los parámetros del modelo de difusión y no sobre la 
señal de difusión tal como la adquieren los escáners. En cambio el trabajo de \citet{Ning2016} combinan la t\'ecnica 
de súper-resolución con el modelo de difusión \textit{compressed-sensing} \citep{Naidoo2015}. A diferencia del primer 
trabajo, utilizan un algoritmo de optimización convexa para reconstruir la imagen con alta calidad. En ambos casos el 
algoritmo depende fuertemente del modelo seleccionado para representar la señal de difusión.

%\subsection{El objetivo}
El objetivo de este trabajo es diseñar, por primera vez, una técnica de transferencia de calidad 
imágenes, para el modelo \textit{Multi-Spherical Diffusion MRI} que toma en cuenta además la dimensión temporal 
\citep{Fick}. Este modelo es el \'unico al momento que utiliza el análisis harmónico para modelar la señal de difusión 
en función del espacio $q$ y del tiempo de difusión. Lo cual es difícil, debido a que encontrar la manera más adecuada 
de muestrear un dominio de semejantes proporciones, con el fin de reconstruir fielmente la señal resulta 
sumamente complicado.
%Debido que como ante cualquier dominio de muestro de semejantes proporciones, encontrar la manera mas adecuada de 
%muestrearlo con el fin de poder de recosntruir fielmente la señal resulta sumamente complicado.
%
%no ahora tiene que pensar también en la mejor manera muestrear q - espacio
%(Con puntos cartesianos , líneas radiales , o cáscaras esféricas como la Fig. 1 , o de otro tipo de muestreo no 
%uniforme ) con el fin de tener medidas de difusión suficientes para reconstruir con firmeza la PEA y evitar la 
%necesidad de cientos de mediciones de difusión. Diferentes esquemas SAMPLING podrían estar mejor adaptados en función 
%de las funciones de base elegidos utilizados en el algoritmo de reconstrucción . De hecho , la base 3D SPF y la base 
%polinomios de Hermite tienen más grados de libertad que nuestra ecuación de Laplace , ya que requiere la fase m , 
%orden ' de la base SH , y el orden polinomio n , mientras que sólo necesitamos la fase y el orden ( ' , metro). Esta 
%flexibilidad adicional podría potencialmente permitir tener un muestreo no uniforme más complejas en el espacio q- . 
%Por otra parte , dicha flexibilidad podría llegar a ser una dificultad adicional para determinar el esquema de 
%muestreo óptima.
Tanto la complejidad espacial, como temporal de 
las operaciones necesarias para trabajar con estos vol\'umenes de datos, representan un desafío computacional, en cuanto 
a desarrollar un algoritmo eficiente que lleve a cabo el realce buscado. Para ello plantearemos el objetivo en términos 
de un problema de optimización convexa, imponiendo además suavidad en la imagen resultante, dado la correlaci\'on que 
existe entre voxeles vecinos. Plantearlo de esta manera nos permitirá sacar ventajas de la naturaleza esparza de la 
señal de difusión. Además dada la heterogeneidad de los tejidos neuronales, usaremos algoritmos de aprendizaje 
automático con el fin de clasificar los tipos de tejidos, para poder hacer un realce más preciso según la región de la 
imagen. Utilizaremos la base de datos \textit{Human Connectome Project} \citep{Sotiropoulos2013125}, la cual posee gran 
cantidad de imágenes dMRI con estándares de 
resolución muy alta.

\clearpage
\bibliographystyle{plainnat}
\bibliography{references.bib}












\end{document}
\grid
