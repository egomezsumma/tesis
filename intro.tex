
\chapter{Introducci\'on}

\section{Imágenes por resonancia magnética (MRI)}
%{\begin{small}%
%\begin{flushright}%
%\it
%There's nothing for me now.
%I want to learn the ways of\\ the Force and become a Jedi like my father. \\
%--Luke Skywalker
%\end{flushright}%
%\end{small}%
%\vspace{.5cm}}


%ESTR:
%[MRI zoom:3][DMRI zoom3]


La Resonancia Magn\'etica es una t\'ecnica útil para generar im\'agenes de 
\'organos internos del cuerpo, sin la necesidad de someter al mismo a un 
procedimiento quir\'urgico. Estas im\'agenes permiten a los m\'edicos hacer distintos tipos de 
diagn\'osticos. La imagen construida por esta t\'ecnica esta fuertemente 
relacionada con la densidad de part\'iculas de hidr\'ogeno en un determinado 
voxel (p\'ixel en tres dimensiones).

Esta t\'ecnica, a grandes rasgos, se basa en los siguientes fen\'omenos f\'isicos. El primero de 
ellos, es 
una propiedad que poseen las part\'iculas de los n\'ucleos de hidr\'ogeno, 
llamada momento angular intrínseco (\textit{spin angular momentun} en ingles). 
Mediante la cual las part\'iculas se encuentran girando en su 
propio eje a una frecuencia determinada. La razón de este movimiento se llama razón 
giromagnética (\textit{gyromagnetic 
ratio} en ingles) y lo denotamos con el símbolo $\gamma$. El valor de esta 
razón es propio de cada sustancia y su unidad en el 
\textit{sistema internacional} es de radianes sobre segundos por tesla. Debido 
a este movimiento, los n\'ucleos se encuentran 
magnetizados y se comportan como un peque\~no im\'an bipolar, los cuales 
propagan cierto campo magn\'etico en una cierta direcci\'on. Es por ello, que se los pueden 
representar como un vector en tres dimensiones. La orientaci\'on del campo la representamos con 
la direcci\'on del vector y la magnitud del campo con longitud del mismo.

El siguiente fen\'omeno, es la consecuencia de aplicar un campo magnético externo $B$ a partículas que se 
encuentran girando en su propio eje. La consecuencia será que estas comenzarán a describir un 
movimiento de precesi\'on alrededor de un eje imaginario paralelo a la direcci\'on del campo 
magnético $B$. La frecuencia de este movimiento, en cada part\'icula será proporcional a $B$ y a su 
razón giromagnética (ecuación \ref{eq:1}).

\begin{equation} 
\label{eq:1}
 w=\gamma  B
\end{equation}


Medir el campo magn\'etico de una \'unica part\'icula (o inclusive de varias) resulta imposible. 
Debido a que, por un lado, el campo magn\'etico es muy d\'ebil como para poder ser captado por las 
maquinarias de hoy en d\'ia. Y la distribuci\'on uniforme de las part\'iculas 
de hidr\'ogeno en los tejido provoca que sus respectivos vectores se cancelen (es decir, que sumen 
una resultante nula). 

Con el fin de poder medir el campo magnético que propagan las partículas, se les aplican una 
secuencia de pulsos de radiofrecuencia junto con campos magnéticos gradientes (i.e. que cambian 
linealmente en el espacio a lo largo de una direcci\'on). Dichos campos magn\'eticos gradientes 
hacen que las part\'iculas se alineen convenientemente y que a distintas posiciones en el espacio, 
posean distintas frecuencias de precesi\'on. La posibilidad de alinearlas causa que la resultante de 
los vectores que representan a las part\'iculas ya no se cancelen. Al girar a distintas 
frecuencias en funci\'on de la posici\'on, ayuda a discriminar de que punto o plano del cuerpo 
proviene la se\~nal que queremos medir.

Ya que podemos saber la frecuencia en la que giran las part\'iculas en cierta posici\'on por la
ecuación de Larmor (ecuación \ref{eq:1}). Y gracias al fen\'omeno f\'isico conocido como 
\textit{resonancia}, podemos generar un pulso radio magnético el cual 
incrementar\'a considerablemente la velocidad de giro a las part\'iculas que 
est\'en girando a la misma frecuencia que dicho pulso. Excitar un conjunto 
de part\'iculas de esta manera en una cierta posici\'on, les aplica una 
gran cantidad de energ\'ia. Y ahora si, con esta energía adicional, el campo 
magn\'etico que propagan los n\'ucleos de hidr\'ogeno excitados puede ser 
captado por una m\'aquina. Con esta se\~nal los escáneres pueden construir las 
im\'agenes MRI.


\section{Resonancia magnética de difusi\'on}


La resonancia magnética de difusi\'on (dMRI del termino en ingles 
\textit{Diffusion MRI}) es la extensi\'on de MRI con el fin de poder adem\'as  
representar la 
difusi\'on de las mol\'eculas del agua dentro de los tejidos. En nuestro caso particularmente 
dentro de la materia blanca del cerebro. Estas im\'agenes proveen valiosa información sobre la 
micro-estructura de los diferentes tipos de tejidos del cerebro (LeBihan and breton 1985). En la 
actualidad, es la única técnica capaz de evaluar 
%de forma no invasiva 
la conectividad neuronal en un cerebro humano vivo y por lo tanto posee un enorme potencial para 
resolver problemas actuales en el área de la neurociencia.

La importancia cl\'inica de las im\'agenes de difusi\'on cobro 
importancia cuando \citet{Moseley1990} reportaron que con dichas im\'agenes es 
posible discriminar entre tejido normal y tejido isqu\'emico. Ninguna 
otra t\'ecnica pod\'ia hacerlo, con la excepci\'on de la tomograf\'ia con 
emisi\'on de positrones. 

Uno de los desafíos importantes de la física y de la química del siglo $xx$, 
fue como cuantificar la difusión de un liquido en un medio. \citet{Hahn1950} se da 
cuenta que la señal MRI era sensible a la difusión. Fue as\'i, que \citet{CarrH.Y.andPurcell1954} 
tuvieron la idea de sensibilizar los escáneres de resonancia magnética al movimiento de las 
part\'iculas de agua, 
modificando la secuencia de campos magn\'eticos gradientes.

Como la difusi\'on de las part\'iculas de agua es influenciada por la estructura del medio en el 
cual difunde, entonces la resonancia magnética puede usarse para entender la 
estructura de los tejidos de una manera no invasiva.  


Casi una d\'ecada despu\'es \citet{Stejskal1965} proponen una nueva secuencia 
llamada \textit{Pulse Gradient Spin Echo} (PSGE). La cual agrega otro
gradiente a la secuencia utilizada para generar las imágenes MRI. Este nuevo 
gradiente $G$ a\~nade velocidad de precesi\'on a los n\'ucleos en un 
determinado momento del experimento por un tiempo $\delta$. Luego se apaga dicho 
campo magn\'etico gradiente y se enciende otro exactamente igual, pero en 
sentido contrario y durante el mismo per\'iodo de tiempo $\delta$. El primer 
gradiente causa que los n\'ucleos que est\'an precediendo se desfasen. 
El desfasaje provoca que una de las tres componentes que posee la resultante de los vectores que 
representan a los n\'ucleos, se cancele (por convenci\'on se coinsidera la componente vertical). Con 
lo cual el escáner percibir\'a una ca\'ida de se\~nal mucho m\'as r\'apida que con el experimento 
sin gradiente $G$. Ahora bien, el segundo gradiente, debería  corregir esa diferencia de fase (ya 
que le aplica el mismo impulso, el mismo tiempo, pero en el sentido coconsidera. Sin embargo no es 
del todo cierto, y es porque las part\'iculas no est\'an quietas. Las mismas se est\'an moviendo 
debido a la difusi\'on. Por lo tanto, la señal global, dada por la suma de los momentos magnéticos 
de todas las particulas, se atenúa debido a la incoherencia en las orientaciones de los momentos mageneticos
individuales. Entonces esa caída más abrupta de la señal nos 
aporta informaci\'on de la difusión en ese punto. Es por ello, que para 
hablar de difusi\'on lo que hacemos es comparar una 
imagen tomada sin gradientes $G$ y una con un gradiente no nulo. Porque la 
diferencia que hay entre ambas es debido a la difusi\'on.



Cabe aclarar que la estructura del cerebro es $anisiotropica$ en relación 
a la difusión de las part\'iculas hidrógeno. Esto quierepartículas el valor 
medido en un determinado voxel depende de la direcci\'on en la que se la mida. 
Es por ello, que para reconstruir los modelos que 
representan la difusi\'on de las partículas de hidrógeno en el cerebro, es 
necesario hacer mediciones en varias direcciones distintas. Inclusive, existen 
modelos como NODDI (ref) que adem\'as de distintas direcciones precisan 
distintas intensidades de G. 

El dominio de muestreo de la señal que da origen a este tipo de imágenes, no es solo 
espacial. Este comprende además la intensidad del gradiente de difusión aplicado, la dirección del mismo y el 
tiempo de difusión. Todos estos parámetros gracias al trabajo de \citet{CALLAGHAN1990177} pueden ser codificados en una única variable 
conocida como $q$ (ecuasión \refecuación Tomar varias mediciones para distintos valores de la variable $q$, se lo denomina 
muestrear el espacio $q$ (o \textit{q-space} del termino en ingles). Como esta variable esta compuesta por varios parámetros, dependiendo el tipo de experimento se 
suele hacer variar la dirección de G (a este conjunto de datos se los denomina $only$-$shell$ del termino en ingles). Otros experimentos,
además hacen variar la intensidad de G (conjunto de datos $multi$-$shell$). Y algunos mas ambiciosos, por la cantidad de datos a manejar, 
ademas varían el tiempo de difusión. Es por ello que decimos que estas imágenes además de resolución espacial, poseen resolución angular y temporal.


\begin{equation} 
\label{eq:3}
 q = \gamma \delta G
\end{equation}

\begin{equation} 
\label{eq:2}
E(q) = \frac{S(q)}{S_0} \hspace{2cm} S_0=S(q=0)
\end{equation}

%\begin{center}
%$E(G, \Delta, \delta) = \frac{S(G, \Delta, \delta)}{S_0}$ 
%$$S_0=S(q=0)$$
%\end{center}

A $E(q)$ la llamamos \textit{atenuaci\'on de la señal},  
$\Delta$ es el tiempo de difusión que es el tiempo entre la activación del primer gradiente y el segundo en 
sentido contrario, el cual describimos en el párrafo anterior. Esta señal nos 
habla \'unicamente de la difusi\'on de las part\'iculas de hidr\'ogeno en la misma 
direcci\'on en la cual se aplico el campo magn\'etico gradiente $G$, en un voxel determinado del 
cerebro. Dicha $G$ es un vector en 3 dimensiones que representa la intensidad 
del campo gradiente y su direcci\'on.


Con el correr de los años una de las lineas de investigación más 
importante en el \'area fue la interpretaci\'on f\'isica de la se\~nal recibida 
por los escáneres de MR.  



\section{Modelos de difusi\'on}
La materia blanca es uno de los varios tipos de tejidos que encontramos en el cerebro. Se sabe que 
la misma esta conformada en su mayor\'ia por axones \citep{Purves2004}. Los axones son una 
prolongaci\'on larga que constituye la morfolog\'ia de las c\'elulas neuronales. 
Luego entender en que direcciones prefiere difundir el agua dentro de la materia 
blanca es \'util para poder entender la estructura de los axones con un estudio \textit{in-vivo} 
(ref). Aunque las part\'iculas de hidr\'ogeno pueden difundir hacia cualquier 
direcci\'on del espacio, su gran mayor\'ia encontrara menos obst\'aculos al difundir en una 
direcci\'on alineada con el ax\'on (o fajo de axones) que se disponga en ese lugar. Con este 
conocimiento a priori de la anatomía de la materia blanca y para obtener detalles más finos sobre 
las características micro-estructurales del tejido cerebral, como por ejemplo el diámetro axonal 
promedio en voxel. Se puede ajustar un modelo matemático a los datos adquiridos. Uno de los 
principales objetivos de los modelos de difusión es recuperar información más detallada sobre la 
orientación de las fibras, haciendo uso del conocimiento a priori que se posee sobre la anatomía de 
los tejidos neuronales. Entre ellos se encuentra el modelo \textit{Diffusion Tensor Imaging} (DTI). 
En el cual se representa la se\~nal de difusión con un tensor de segundo orden \citep{Basser1994}. 
También existen modelos más complejos como por ejemplo MAPL \citep{Fick2016365}. El mismo pertenece 
a una familia de modelos que parte del enfoque de representar la señal de difusión con 
herramientas de análisis armónico.

\section{Problema y objetivo}
Los estándares de resolución de hoy en día de dichas imágenes, permiten hacer diversos análisis sobre conjuntos de 
fibras de mediano y gran tamaño. Sin embargo, estos análisis pierden precisión en regiones donde los grupos de axones 
son muy pequeñas en comparación al tamaño del voxel (4 ordenes de magnitud más pequeños). Consecuentemente, existe la 
necesidad de incrementar tanto la resolución espacial, angular y temporal de las imágenes dMRI. Dado que el dominio de 
muestreo de la señal difusión, no es solo espacial como ya vimos en la sección anterior. La baja relación señal ruido y el tiempo de adquisición de las 
imágenes dMRI son unos de los principales factores que limitan la resolución de las mismas. %Los limites para obtener imágenes con alta resolución 
%están relacionados con 
%el tiempo de adquisición de las mismas y la relación señal ruido (la cual decrece 
%con el tamaño voxel). 
Hoy en día, existen escáneres que pueden producir imágenes con muy alta resolución. Pero los mismos son muy costosos e 
imprácticos para escenarios clínicos. El objetivo de este trabajo es aumentar la resolución de las imágenes de difusión.


